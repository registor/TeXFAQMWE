% 摘录自https://tex.stackexchange.com/questions/515625/draw-derivative-tangent-at-specific-point-in-tikz
\documentclass[tikz,border=3mm]{standalone}

\usepackage{ctex}% 支持中文

\usetikzlibrary{intersections, calc}

\begin{document}
\begin{tikzpicture}[
  domain=1.5:4.1,
  samples=101,
  pics/tangent at/.style={
    code={
      \path[name path=l] (#1 - 1pt,0|-current bounding box.south) -- (#1 - 1pt,0|-current bounding box.north);
      \path[name path=r] (#1 + 1pt,0|-current bounding box.south) -- (#1 + 1pt,0|-current bounding box.north);
      \path[pic actions,
        %name path=t,
        name intersections={of=\pgfkeysvalueof{/tikz/tangent pic/graph name} and l,by=li},
        name intersections={of=\pgfkeysvalueof{/tikz/tangent pic/graph name} and r,by=ri}
        ] (li) -- (ri);
    }
  },  
  tangent pic/.cd,
  graph name/.initial=graph
  ]

  % 绘制命名曲线
  \draw[blue,thick,name path=graph] plot (\x,{\x^3-7.7*\x^2+19.2*\x-15.5}) node[right] {$f(\mathrm{x})$};

  % 绘制辅助网格线
  % \draw[very thin,gray] (-1,-3) grid (10,10);
  % 绘制x轴和y轴
  \draw[->, name path=x] (0,0) -- (5,0) node[right] {$\mathrm{x}$};
  \draw[->] (0,-0.5) -- (0,3.5) node[left] {$\mathrm{y}$};

  % 绘制真值点
  \fill[fill=black, name intersections={of=graph and x}]
  (intersection-3) circle[radius = 0.015cm] node[above, font=\tiny,xshift=-0.2cm]{$x^*$};

  % 绘制切线
  \newdimen\dx
  
  \fill[fill = black] (4, 0) coordinate (xs) circle[radius = 0.015cm] node[below, font = \tiny] {$x_{1}F$};  
  \pgfpointanchor{xs}{center}
  \pgfextractx{\dx}{xs}
  \path pic{tangent at=\dx};
  \path[name path=t](ri)--($(ri)!10!(li)$);
  \path[name path=v] (\dx, 0|-current bounding box.south) -- (\dx, 0|-current bounding box.north);
  
  \fill[fill = black,name intersections={of=t and x, by=tx}] (tx) circle[radius = 0.015cm] node[below,
  font = \tiny] {$x_{2}$};
  \fill[fill = black,name intersections={of=v and graph, by=fx}] (fx) circle[radius = 0.015cm] node[left,
  font = \tiny] {$f(x_{1})$};
  \draw[red, dashed] (fx)--(tx) (fx)--(xs);

  \pgfpointanchor{tx}{center}
  \pgfextractx{\dx}{tx}

  % \path[draw,green,name path=l] (\dx - 1pt,0|-current bounding
  % box.south) -- (\dx - 1pt,0|-current bounding box.north);
  % \path[draw,green,name path=r] (\dx + 1pt,0|-current bounding
  % box.south) -- (\dx + 1pt,0|-current bounding box.north);
 
  \path pic{ tangent at=\dx};
  \path[name path=t](ri)--($(ri)!10!(li)$);
  \path[name path=v] (\dx, 0|-current bounding box.south) -- (\dx, 0|-current bounding box.north);
  
  %\fill[fill = black] (\dx, 0) coordinate (xs) circle[radius = 0.015cm] node[below, font = \tiny] {$x_{3}$}; 
  \fill[fill = black, name intersections={of=t and x, by=tx}] (tx) circle[radius = 0.015cm] node[below,
  font = \tiny] {$x_{3}$};
  \fill[fill = black, name intersections={of=v and graph, by=fx}] (fx) circle[radius = 0.015cm] node[left,
  font = \tiny] {$f(x_{2})$};

  \draw[green, dashed] (fx)--(tx) (fx)--(\dx, 0);

  \path[draw, black] (4, 0|-current bounding box.south) -- (4,0|-current bounding box.north);
  

  
\end{tikzpicture}
\end{document}
%%% Local Variables:
%%% mode: latex
%%% TeX-master: t
%%% End:
