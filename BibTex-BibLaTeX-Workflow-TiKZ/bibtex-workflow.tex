% 使用ctexart文档类(用XeLaTeX编译,直接支持中文)
\documentclass[12pt]{standalone}
\usepackage{ctex}
% 部分latex的Logo
\usepackage{xltxtra}

\usepackage{tikz-flowchart}
% \usepackage{tikz}
% \usetikzlibrary{positioning}
% \usetikzlibrary{calc}
% \usetikzlibrary{arrows.meta}

% 各参数默认值
\flowchartset{
  free color = green, % 自由连线颜色(默认取green)
  norm color = blue, % 常规连线颜色(默认取blue)
  cong color = red, % 关联连线颜色(默认取red)
  proc fill color = red!10, % 顺序处理框填充颜色(默认取白色)
  proc text width = 4em, % 顺序处理框宽度(默认取8em)
  chain direction = right, % 结点自动布置方向(默认取below)
  minimum node distance = 6mm, % 最小结点间距(默认取6mm)
  maximum node distance = 10mm, % 最大结点间距(默认取60mm)
  flow line width = \pgflinewidth, % 各类流程线线条宽度(默认取当前线条宽度)
  stealth length = 1.5mm, % 箭头长度(默认取1.5mm)
  stealth width = 1.0mm, % 箭头宽度(默认取1.0mm)
}

\tikzset{cirnum/.style={midway, circle, fill=black, inner
    sep=0pt, minimum size=3pt, white, yshift=0.35em}} 

\begin{document} %在document环境中撰写文档

\begin{tikzpicture}[font=\scriptsize]
  %\node[proc, fill=blue!30](src){.tex源文档};
  \node[proc, fill=yellow!30](tex){{\LaTeX}引擎};
  \node[proc, join](nobibpdf){无文献PDF};
  \node[proc, below=1.5 of nobibpdf, text width = 4.8em, fill = violet!40, shift = {(-1.0, 0.0)}](bib){.bib文献数据库};
  \node[proc, right=2.5 of bib, text width = 4.6em, fill = violet!40](bst){.bst文献样式};
  \node[proc, below=3.0 of tex, text width = 4.6em, fill=green!30, dashed](aux){.aux(辅助文件)};
  \node[proc, right=3.3 of aux, text width = 4.5em, fill=yellow!30](bibtex){bibtex工具};
  \node[proc, right=3.0 of bibtex, text width = 4.5em, fill=green!30, dashed](bbl){.bbl文献列表};
  \node[proc, below=2.5 of aux, fill=yellow!30](btex){{\LaTeX}引擎};  
  \node[proc, join, text width = 4.5em](norefpdf){无引用PDF};
  \node[proc, left = 3.2 of btex, fill=blue!30](src){.tex源文档};
  \node[proc, below=1.5 of btex, text width = 4.6em, fill=green!30, dashed](baux){.aux(辅助文件)};
  \node[proc, below=1.5 of baux, fill=yellow!30](bbtex){{\LaTeX}引擎};
  \node[proc, join, text width = 4.5em, fill=red!60](pdf){最终PDF};

  \path (tex.east) to node[cirnum] {\tiny 1} (nobibpdf);
  \path (tex.east) to node[cirnum] {\tiny 1} (nobibpdf);
  \path (btex.east) to node[cirnum] {\tiny 3} (norefpdf);
  \path (bbtex.east) to node[cirnum] {\tiny 4} (pdf);
  
  \draw[norm] (src) -- node[cirnum, below]{\tiny 3} (btex);
  \draw[norm] (src) edge [bend left] node[cirnum, right]{\tiny 1} (tex);
  \draw[norm] (src) edge [bend right] node[cirnum, right]{\tiny 4} (bbtex);

  \draw[free, dashed, green!80] (tex.south) -- node[cirnum, fill=lightgray] {\tiny 1}(aux.north);
  \draw[free, dashed, red!40] (aux) -- node[cirnum, fill=lightgray] {\tiny 2}(bibtex);
  \draw[free, dashed, green!80] (bibtex) -- node[cirnum, fill=lightgray] {\tiny 2}(bbl); 
  \draw[norm] (bib.south) -- node[cirnum, below]{\tiny 2} (bibtex.north);
  \draw[norm] (bst.south) -- node[cirnum, below]{\tiny 2} (bibtex.north);
  
  \draw[free, dashed, red!40] (aux) -- node[cirnum, fill=lightgray] {\tiny 3}(btex);
  \draw[free, dashed, red!40] (bbl.south) -- node[cirnum, fill=lightgray] {\tiny 3}(btex.north);

  \draw[free, dashed, green!80] (btex.south) -- node[cirnum, fill=lightgray] {\tiny 3}(baux.north);
  \draw[free, dashed, red!40] (baux.south) -- node[cirnum, fill=lightgray] {\tiny 4}(bbtex.north); 
  \draw[free, dashed, red!40] (bbl.south) to [out=-90, in = 45] 
  node[cirnum, fill=lightgray] {\tiny 4}(bbtex.north east); 
\end{tikzpicture}

\end{document}

%%% Local Variables:
%%% mode: latex
%%% TeX-master: t
%%% End:
