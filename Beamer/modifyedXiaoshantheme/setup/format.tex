\makeatletter
% pretend to already have loaded beamerfontthememetropolis
\@namedef{ver@beamerfontthememetropolis.sty}{9999/99/99}

% Beamer theme
\usetheme{Xiaoshan}
\usefonttheme{professionalfonts}

% Beamer settings
\metroset{progressbar=none}
\setbeamerfont{title}{size=\huge, series=\bfseries}
\setbeamerfont*{subtitle}{size=\large, shape=\itshape}
\setbeamerfont{section title}{size=\Large, series=\bfseries}
\setbeamerfont{frametitle}{size=\large, series=\bfseries}
\setbeamerfont{caption}{size=\footnotesize, series=\bfseries}
\setbeamerfont{footnote}{size=\tiny}
\setbeamerfont{alerted text}{series=\bfseries}
\addtobeamertemplate{institute}{\raggedleft}{}
\setbeamertemplate{title}{%
  \raggedleft
  \linespread{1.0}%
  \inserttitle
  \hspace*{1.2cm}\par
  \vspace*{0.5em}}
\setbeamertemplate{subtitle}{%
  \raggedleft
  \insertsubtitle
  \hspace*{1.2cm}\par
  \vspace*{0.5em}}
\setbeamertemplate{title page}{
  \begin{minipage}[b]{\textwidth}
    \usebeamertemplate*{title graphic}\vfill
    \usebeamertemplate*{title}
    \usebeamertemplate*{subtitle}
    \usebeamertemplate*{title separator}
    \usebeamertemplate*{author}
    \usebeamertemplate*{date}
    \usebeamertemplate*{institute}
    \vfill
  \end{minipage}}
\setbeamertemplate{frame numbering}{{\insertframenumber/\inserttotalframenumber}}%\zhnumber[style=Financial]
\setbeamertemplate{itemize/enumerate subbody begin}{\footnotesize}
\setbeamertemplate{caption}{\parbox{\textwidth}{\centering\insertcaption}\par}
\setbeamertemplate{bibliography item}[text]

% 设置帧标题格式
\setbeamertemplate{frametitle}{
    \ifbeamercolorempty[bg]{frametitle}{}{\nointerlineskip}%
    \@tempdima=\textwidth%
    \advance\@tempdima by\beamer@leftmargin%
    \advance\@tempdima by\beamer@rightmargin%
    %\hspace*{1cm} %%%%%%%%%%%%% For example insert shift to right
    \begin{beamercolorbox}[sep=0.3cm,wd=\the\@tempdima]{frametitle}
        \usebeamerfont{frametitle}%
        \vbox{}\vskip-1ex%
        \if@tempswa\else\csname beamer@ftecenter\endcsname\fi%
        \strut\insertframesubtitle\hfill\strut%\par%
        {%
            \ifx\insertframetitle\@empty%
            \else%
            {\usebeamerfont{frametitle}\usebeamercolor[fg]{frametitle}|\insertframetitle\strut}%\par}%
            \fi
        }%
        \vskip-1ex%
        \if@tempswa\else\vskip-.3cm\fi% set inside beamercolorbox... evil here...
    \end{beamercolorbox}%
}

% % 设置脚注格式
% %% put a piece of text in bottom right of a beamer slide with command \br{}
% %% adapted from http://tex.stackexchange.com/questions/89539/absolute-positioning-in-beamer-when-using-show-notes-on-second-screen-text-on
% \setbeamercolor{framesource}{fg=gray}
% \setbeamerfont{framesource}{size=\footnotesize}
% \makeatletter
% \renewcommand{\footnoterule}{}
% \define@key{beamerfootnote}{nonumber}[true]{\edef\beamer@footarg{0}\def\@makefnmark{}}% have to set a number in \beamer@footarg, then it won't be automatically generated one. setting \@makefnmark to be empty means the number isn't printed. but only use this in a group else it affects following footnotes!
% % instead of 'nonumber', could just use 0 as an optional argument to \footnote, but OP reports that keyval complains about this in some situations
% % http://tex.stackexchange.com/questions/89539
% \newcommand{\br}[1]{{\footnote[nonumber]{%
%     \begin{beamercolorbox}[right,wd=\dimexpr\hsize-1.8em\relax]{framesource}
%       \setlength\topsep{0pt}\rule{0.65\textwidth}{0.4mm}\vspace{0.5em}\\[-6pt]
%       \usebeamerfont{framesource}\usebeamercolor[fg]{framesource}\itshape{#1}
%     \end{beamercolorbox}}}}
% \setlength{\footnotesep}{0cm}%\footnotesep is the space between footnotes (generated with a \rule)
% \makeatother

% \setbeamertemplate{footline}
% {
%   \leavevmode%
%   \hbox{%
%   \begin{beamercolorbox}[wd=.333333\paperwidth,ht=2.25ex,dp=1ex,center]{author in head/foot}%
%     \usebeamerfont{author in head/foot}\insertsection
%   \end{beamercolorbox}%
%   \begin{beamercolorbox}[wd=.333333\paperwidth,ht=2.25ex,dp=1ex,center]{title in head/foot}%
%     \usebeamerfont{title in head/foot}\insertsubsection
%   \end{beamercolorbox}%
%   \begin{beamercolorbox}[wd=.333333\paperwidth,ht=2.25ex,dp=1ex,right]{date in head/foot}%
%     \usebeamerfont{date in head/foot}\insertshortdate{}\hspace*{2em}
%     \insertframenumber{} / \inserttotalframenumber\hspace*{2ex} 
%   \end{beamercolorbox}}%
%   \vskip0pt%
% }

\setbeamertemplate{footline}
{
    \leavevmode%
    \hbox{%
        \begin{beamercolorbox}[wd=.25\paperwidth,ht=2.25ex,dp=1ex,center]{frametitle}%{author in head/foot}%
            \usebeamerfont{author in head/foot}\insertauthor{}(\insertshortauthor{})
        \end{beamercolorbox}%
        \begin{beamercolorbox}[wd=.5\paperwidth,ht=2.25ex,dp=1ex,center]{frametitle}%{title in head/foot}%
            \usebeamerfont{title in head/foot}\insertshorttitle
        \end{beamercolorbox}%
        \begin{beamercolorbox}[wd=.25\paperwidth,ht=2.25ex,dp=1ex,right]{frametitle}%{date in head/foot}%
            \usebeamerfont{date in head/foot}\insertshortinstitute{}\hspace*{2em}
            \insertframenumber{}/\inserttotalframenumber{}\hspace*{2ex} 
        \end{beamercolorbox}}%
        \vskip0pt%
    }

%改变脚注的符号
\setbeamerfont{footnote}{size=\zihao{7}} % 改变脚注字号

\makeatletter
\def\@fnsymbol#1{\ensuremath{\ifcase#1\or *\or \dagger\or \ddagger\or
   \mathsection\or \mathparagraph\or \|\or **\or \dagger\dagger
   \or \ddagger\ddagger \else\@ctrerr\fi}}
\makeatother
\renewcommand{\thefootnote}{\fnsymbol{footnote}}



% Colors
\colorlet{keyword}{松花绿}
\colorlet{comment}{漆黑!50}
\colorlet{texcs}{酡红}
\colorlet{emph1}{靛蓝}
\colorlet{emph2}{琥珀}
\colorlet{inline}{玄色}

% Hack
% Use small caps for LaTeX symbol
\DeclareRobustCommand{\LaTeX}{%
  L\kern-.3em%
  \raisebox{.2em}{\textsc{a}}\kern-.14em%
  \TeX}
% Compatibility with unicode-math
\DeclareRobustCommand{\LaTeXe}{%
  \LaTeX\kern.15em2%
  \hbox{%
    \if b\expandafter\@car\f@series\@nil
      $_{\textstyle\symbf{\varepsilon}}$%
    \else
      $_{\textstyle\varepsilon}$%
    \fi}}
% PoZheHao, see https://github.com/CTeX-org/ctex-kit/issues/382
\ExplSyntaxOn
\xeCJK_new_class:n { PoZheHao }
\__xeCJK_save_CJK_class:n { PoZheHao }
\xeCJK_declare_char_class:nn { PoZheHao } { "2014 }
\seq_map_inline:Nn \g__xeCJK_class_seq
  {
    \str_if_eq:nnF {#1} { PoZheHao }
      {
        \xeCJK_copy_inter_class_toks:nnnn { PoZheHao } {#1} { FullRight } {#1}
        \xeCJK_copy_inter_class_toks:nnnn {#1} { PoZheHao } {#1} { FullRight }
      }
  }
\ExplSyntaxOff

% Macros
\newcommand\link[1]{\href{#1}{\faLink}}
\newcommand\CASE[1]{{\addfontfeatures{Letters=Uppercase}#1}}
\newcommand\jatext[1]{{\addCJKfontfeatures{Language=Japanese}#1}}
\newcommand\zhparen[1]{(\raisebox{0.1ex}{#1})}
\newcommand\enparen[1]{\CASE{(}#1\CASE{)}}
\newcommand\pkg[1]{\texttt{#1}}
\newcommand\usv[1]{\texttt{U+#1}}
\newcommand\kbd[1]{{\LibertinusKey#1}}
\DeclareRobustCommand{\nonumberfootnote}[2][]{%
  \let\thefootnote\relax
  \footnotetext#1{#2}}
\renewcommand*\footnoterule{}

\newcommand\XeTeX{\hologo{XeTeX}}
\newcommand\pdfTeX{\hologo{pdfTeX}}
\newcommand\LuaTeX{\hologo{LuaTeX}}
\newcommand\XeLaTeX{\hologo{Xe}\kern-.13em\LaTeX{}}
\newcommand\pdfLaTeX{pdf\LaTeX{}}
\newcommand\LuaLaTeX{Lua\LaTeX{}}
\newcommand\BibTeX{\hologo{BibTeX}}
\makeatother

%\setminted{breaklines=true, breakautoindent=false}
\newmintinline{tex}{fontsize=\normalsize}
\newmintinline[texinlinett]{tex}{fontsize=\normalsize,escapeinside=||}
\newminted{tex}{bgcolor=yellow!20, frame=lines}
\newminted[texcodett]{tex}{bgcolor=yellow!20, frame=lines, escapeinside=||}
\newminted[shell]{sh}{autogobble,fontsize=\small,frame=lines}
\newmintedfile{tex}{bgcolor=yellow!20, fontsize=\footnotesize, frame=lines}

%% 自定义相关的名称宏命令
%% ==================================================
%% \newcommand{\yourcommand}[参数个数]{内容}
% 西北农林科技大学各单位名称
\newcommand{\nwsuaf}{西北农林科技大学}
\newcommand{\cie}{信息工程学院}
\newcommand{\cs}{计算机科学系}

\newcommand\latex{{\fontfamily{cmr}\selectfont \LaTeX}}
\newcommand\msoffice{{\rmfamily MS Office}}
\newcommand\msofficepdf{\texorpdfstring{\msoffice{}}{MS Office}}

\newcommand\wysiwym{\textsc{WYSIWYM}---所想即所得}
\newcommand\wysiwyg{\textsc{WYSIWYG}---所见即所得}

\colorlet{msofficecolour}{red!90!white}
\colorlet{latexcolour}{green!90!black}

\newcommand\latexc{\textcolor{white}{\latex}}
\newcommand\msofficec{\textcolor{white}{\msoffice}}

\newcommand*\TeXLive{T\kern -.1667em\lower .5ex\hbox {E}\kern
  -.025emX\,Live}

\newcommand\tlive[1][2018]{
  \begin{tikzpicture}[x=1pt,y=1pt,inner sep=0pt,outer sep=0pt]
    \fill [tlblue] (0,0) rectangle (567,160);
    \node [white] at (29.7,33.8) [anchor=south west]
    {\scalebox{10}{\bfseries\TeXLive\~ #1}};
    \node at (388,9) [anchor=south west] {\includegraphics[width=15em]{tl-lion}};
    % \node [anchor=south west] {\includegraphics[height=16em]{logo}};
  \end{tikzpicture}%
}

% ==================================================
\newcommand{\nwafuthesis}{%
  \makebox{\rmfamily%
    N\hspace{-0.2ex}\raisebox{-0.5ex}{W}\raisebox{0.5ex}{\hspace{-0.2ex}\textsc{afu}}\hspace{0.3ex}%
    \textsc{Thesis}}}          

%% 定义自动扩展垂直间距的命令\stretchon和\stretchoff
%% ==================================================
\def\itemsymbol{$\blacktriangleright$}
\let\svpar\par
\let\svitemize\itemize
\let\svenditemize\enditemize
\let\svitem\item
\let\svcenter\center
\let\svendcenter\endcenter
\let\svcolumn\column
\let\svendcolumn\endcolumn
\def\newitem{\renewcommand\item[1][\itemsymbol]{\vfill\svitem[##1]}}%
\def\newpar{\def\par{\svpar\vfill}}%
\newcommand\stretchon{%
  \newpar%
  \renewcommand\item[1][\itemsymbol]{\svitem[##1]\newitem}%
  \renewenvironment{itemize}%
    {\svitemize}{\svenditemize\newpar\par}%
  \renewenvironment{center}%
    {\svcenter\newpar}{\svendcenter\newpar}%
  \renewenvironment{column}[2]%
    {\svcolumn{##1}\setlength{\parskip}{\columnskip}##2}%
    {\svendcolumn\vspace{\columnskip}}%
}
\newcommand\stretchoff{%
  \let\par\svpar%
  \let\item\svitem%
  \let\itemize\svitemize%
  \let\enditemize\svenditemize%
  \let\center\svcenter%
  \let\endcenter\svendcenter%
  \let\column\svcolumn%
  \let\endcolumn\svendcolumn%
}

%% 签署春秋学期日期命令
\newcommand{\tomonth}{
  \the\year 年\the\month 月
}


\newcommand{\tomonthen}{
  \ifcase\the\month
  \or January%
  \or February%
  \or March%
  \or April%
  \or May%
  \or June%
  \or July%
  \or August%
  \or September%
  \or October%
  \or November%
  \or December%
  \fi, \the\year
}

\newcommand{\tosemester}{
  \the\year 年\ 
  \ifcase\the\month
  \or 秋%
  \or 春%
  \or 春%
  \or 春%
  \or 春%
  \or 春%
  \or 春%
  \or 夏%
  \or 秋%
  \or 秋%
  \or 秋%
  \or 秋%
  \fi 
}

\newcommand{\tosemesteren}{  
  \ifcase\the\month
  \or Autumn%
  \or Spring%
  \or Spring%
  \or Spring%
  \or Spring%
  \or Spring%
  \or Summer%
  \or Autumn%
  \or Autumn%
  \or Autumn%
  \or Autumn%
  \or Autumn%
  \fi, \the\year
}

% TiKz绘图设置
\usetikzlibrary{arrows}
% TikZ宏包扩展
\usetikzlibrary{positioning}
\usetikzlibrary{decorations}
% pgfplots设置
\pgfplotsset{compat=newest, compat/show suggested version=false}

%% 设置绘制目录结构的宏参数
\definecolor{folderbg}{RGB}{124,166,198}
\definecolor{folderborder}{RGB}{110,144,169}
\newlength\Size
\setlength\Size{4pt}
\tikzset{%
  folder/.pic={%
    \filldraw [draw=folderborder, top color=folderbg!50, bottom color=folderbg] (-1.05*\Size,0.2\Size+5pt) rectangle ++(.75*\Size,-0.2\Size-5pt);
    \filldraw [draw=folderborder, top color=folderbg!50, bottom color=folderbg] (-1.15*\Size,-\Size) rectangle (1.15*\Size,\Size);
  },
  file/.pic={%
    \filldraw [draw=folderborder, top color=folderbg!5, bottom color=folderbg!10] (-\Size,.4*\Size+5pt) coordinate (a) |- (\Size,-1.2*\Size) coordinate (b) -- ++(0,1.6*\Size) coordinate (c) -- ++(-5pt,5pt) coordinate (d) -- cycle (d) |- (c) ;
  },
}
\forestset{%
  declare autowrapped toks={pic me}{},
  declare boolean register={pic root},
  pic root=0,
  pic dir tree/.style={%
    for tree={%
      folder,
      %font=\ttfamily,
      grow'=0,
      s sep=1.0pt,
      font=\small \sffamily,
      %fit=band,
      %ysep = 1.0pt,
      inner ysep = 2.6pt,
    },
    before typesetting nodes={%
      for tree={%
        edge label+/.option={pic me},
      },
      if pic root={
        tikz+={
          \pic at ([xshift=\Size].west) {folder};
        },
        align={l}
      }{},
    },
  },
  pic me set/.code n args=2{%
    \forestset{%
      #1/.style={%
        inner xsep=2\Size,
        pic me={pic {#2}},
      }
    }
  },
  pic me set={directory}{folder},
  pic me set={file}{file},  
}
%% ==================================================

%%%%%%%%%%%%%%%%%%%%%%%%%%%%%%%%%%%%%%%%%%%%%%%%%%%%%%%%%%%%%%%%%%%%%%
% LaTeX Overlay Generator - Annotated Figures v0.0.2
% Created with http://ff.cx/latex-overlay-generator/
% If this generator saves you time, consider donating 5,- EUR! :-)
%%%%%%%%%%%%%%%%%%%%%%%%%%%%%%%%%%%%%%%%%%%%%%%%%%%%%%%%%%%%%%%%%%%%%%
%                         #1          #2       #3         #4           #5          #6            #7           #8
%\annotatedFigureBox{bottom-left}{top-right}{label}{label-position}{box-color}{label-color}{border-color}{text-color}
\newcommand*\annotatedFigureBoxCustom[8]{\draw[#5,thick,rounded corners] (#1) rectangle (#2);\node at (#4) [fill=#6,thick,shape=circle,draw=#7,inner sep=2pt,font=\sffamily,text=#8] {\textbf{#3}};}
\newcommand*\annotatedFigureBoxLabel[4]{\annotatedFigureBoxCustom{#1}{#2}{#3}{#4}{red}{white}{black}{black}}
\newcommand*\annotatedFigureBox[3]{\draw[#3,thick,rounded corners] (#1) rectangle (#2);}
\newenvironment {annotatedFigure}[1]{\centering\begin{tikzpicture}\node[anchor=south west,inner sep=0] (image) at (0,0) { #1};\begin{scope}[x={(image.south east)},y={(image.north west)}]}{\end{scope}\end{tikzpicture}}
%%%%%%%%%%%%%%%%%%%%%%%%%%%%%%%%%%%%%%%%%%%%%%%%%%%%%%%%%%%%%%%%%%%%%%

% 动态改变menukeys宏包的win/mac样式
\makeatletter
\def\setmenukeyswin{\def\tw@mk@os{win}}
\def\setmenukeysmac{\def\tw@mk@os{mac}}
\makeatother

% 插图路径设置
% ==================================================
\graphicspath{{figures/}}%图片所在的目录
% ==================================================

%%% Local Variables: 
%%% mode: latex
%%% TeX-master: "../main.tex"
%%% End: 
