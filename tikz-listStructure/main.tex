\documentclass[12pt, border = 8pt, varwidth, convert]{standalone}

% 中文支持
\usepackage{ctex}

% TikZ
\usepackage{tikz}
\usetikzlibrary{calc}% 计算库
\usetikzlibrary{arrows.meta}% 箭头库
\usetikzlibrary{shapes}
%%%%%%% 绘制链表数据结构需要的TikZ样式 %%%%%%%%
\tikzstyle{ptr}  = [draw, -{Stealth[scale=1.0]}, blue]
\tikzstyle{head} = [rectangle, draw, text height=3mm, text width=3mm,
                    text centered, node distance=3cm, inner sep=0pt]
\tikzstyle{data} = [rectangle split, rectangle split parts=2, draw,
                    text centered, minimum height=3em]
\newcommand{\data}{
  data \nodepart{second}
  \phantom{\texttt{NULL}}
}
%%%%%%%%%%%%%%%%%%%%%%%%%%%%%%%%%%%%%%%%%%%

\begin{document} %在document环境中撰写文档
% 缩放比例
\def\scaleratio{1.0}
\begin{figure}[htp]
  \centering
  \begin{tikzpicture}[node distance=2cm, auto, scale=\scaleratio,
    every node/.style={scale=\scaleratio}]
    \node[head, label=below:phead] (head) {};
    \node[data, right of=head] (A) {\data};
    \node[above of=A,node distance=0.5cm,label=above:a1] (a1){};
    \node[data, right of=A] (B) {\data};
    \node[above of=B,node distance=0.5cm,label=above:a2] (a2){};
    \node[data, right of=B] (C) {\data};
    \node[above of=C,node distance=0.5cm,label=above:a3] (a3){};
    \node[data, right of=C] (D) {\data};
    \node[above of=D,node distance=0.5cm,label=above:a4] (a4){};
    \node[data, right of=D] (E) {\data};
    \node[above of=E,node distance=0.5cm,label=above:a5] (a5){};
    \node[data, right of=E] (last) {data \nodepart{second} \texttt{NULL}};
    \node[above of=last,node distance=0.5cm,label=above:a6] (a6){};

    \draw[fill] (head.center) circle (0.05);

    \path[ptr] (head.center) --++(right:7.5mm) |- (A.text west);
    \draw[fill] ($(A.south)!0.5!(A.text split)$) circle (0.05);
    \draw[ptr] ($(A.south)!0.5!(A.text split)$) --++(right:10mm) |- (B.text west);
    \draw[fill] ($(B.south)!0.5!(B.text split)$) circle (0.05);
    \draw[ptr] ($(B.south)!0.5!(B.text split)$) --++(right:10mm) |- (C.text west);
    \draw[fill] ($(C.south)!0.5!(C.text split)$) circle (0.05);
    \draw[ptr] ($(C.south)!0.5!(C.text split)$) --++(right:10mm) |- (D.text west);
    \draw[fill] ($(D.south)!0.5!(D.text split)$) circle (0.05);
    \draw[ptr] ($(D.south)!0.5!(D.text split)$) --++(right:10mm) |- (E.text west);
    \draw[fill] ($(E.south)!0.5!(E.text split)$) circle (0.05);
    \draw[ptr] ($(E.south)!0.5!(E.text split)$) --++(right:10mm) |- (last.text west);
  \end{tikzpicture}
  \caption{链表结构}
  \label{fig:list01}
\end{figure}

\begin{figure}[htp]
  \centering
  \begin{tikzpicture}[node distance=2cm, auto, scale=\scaleratio,
    every node/.style={scale=\scaleratio}]
    \node[head, label=below:phead] (head) {};
    \node[data, right of=head] (A) {\data};
    \node[above of=A,node distance=0.5cm,label=above:a1] (a1){};
    \node[data, right of=A] (B) {\data};
    \node[above of=B,node distance=0.5cm,label=above:a2] (a2){};
    \node[data, right of=B] (C) {\data};
    \node[above of=C,node distance=0.5cm,label=above:a3] (a3){};
    \node[data, below of=C, xshift=0.8cm, yshift=0.5cm] (CI) {\data};
    \node[above of=CI,node distance=0.5cm,label=above:a7] (a7){};        
    \node[data, right of=C, xshift=0.1cm] (D) {\data};%, xshift=1.5cm
    \node[above of=D,node distance=0.5cm,label=above:a4] (a4){};
    \node[data, right of=D] (E) {\data};
    \node[above of=E,node distance=0.5cm,label=above:a5] (a5){};
    \node[data, right of=E] (last) {data \nodepart{second} \texttt{NULL}};
    \node[above of=last,node distance=0.5cm,label=above:a6] (a6){};

    \draw[fill] (head.center) circle (0.05);

    \path[ptr] (head.center) --++(right:7.5mm) |- (A.text west);
    \draw[fill] ($(A.south)!0.5!(A.text split)$) circle (0.05);
    \draw[ptr] ($(A.south)!0.5!(A.text split)$) --++(right:10mm) |- (B.text west);
    \draw[fill] ($(B.south)!0.5!(B.text split)$) circle (0.05);
    \draw[ptr] ($(B.south)!0.5!(B.text split)$) --++(right:10mm) |- (C.text west);
    \draw[fill] ($(C.south)!0.5!(C.text split)$) circle (0.05);
        %\draw[ptr] ($(C.south)!0.5!(C.text split)$) --++(right:10mm) |- (CI.text west);
    \draw[ptr] ($(C.south)!0.5!(C.text split)$) |- (CI.text west);
    \draw[fill] ($(CI.south)!0.5!(CI.text split)$) circle (0.05);
    \draw[ptr] ($(CI.south)!0.5!(CI.text split)$) --++(right:6mm) |- (D.text west);
    \draw[fill] ($(D.south)!0.5!(D.text split)$) circle (0.05);
    \draw[ptr] ($(D.south)!0.5!(D.text split)$) --++(right:10mm) |- (E.text west);
    \draw[fill] ($(E.south)!0.5!(E.text split)$) circle (0.05);
    \draw[ptr] ($(E.south)!0.5!(E.text split)$) --++(right:10mm) |- (last.text west);
  \end{tikzpicture}
  \caption{添加链表节点}
  \label{fig:list02}
\end{figure}

\begin{figure}[htp]
  \centering
  \begin{tikzpicture}[node distance=2cm, auto, scale=\scaleratio,
    every node/.style={scale=\scaleratio}]

    \node[head, label=below:phead] (head) {};
    \node[data, right of=head] (A) {\data};
    \node[above of=A,node distance=0.5cm,label=above:a1] (a1){};
    \node[data, right of=A] (B) {\data};
    \node[above of=B,node distance=0.5cm,label=above:a2] (a2){};
    \node[data, right of=B] (C) {\data};
    \node[above of=C,node distance=0.5cm,label=above:a3] (a3){};
    \node[data, below of=C, xshift=0.8cm, yshift=0.5cm] (CI) {\data};
    \node[above of=CI,node distance=0.5cm,label=above:a7] (a7){};        
    \node[data, right of=C, xshift=0.1cm] (D) {\data};
    \node[above of=D,node distance=0.5cm,label=above:a4] (a4){};
    \node[data, right of=D] (E) {data \nodepart{second} \texttt{NULL}};
    \node[above of=E,node distance=0.5cm,label=above:a5] (a5){};
    \node[data, right of=E] (last) {data \nodepart{second} \texttt{NULL}};
    \node[above of=last,node distance=0.5cm,label=above:a6] (a6){};

    \draw[fill] (head.center) circle (0.05);

    \path[ptr] (head.center) --++(right:7.5mm) |- (A.text west);
    \draw[fill] ($(A.south)!0.5!(A.text split)$) circle (0.05);
    \draw[ptr] ($(A.south)!0.5!(A.text split)$) --++(right:10mm) |- (B.text west);
    \draw[fill] ($(B.south)!0.5!(B.text split)$) circle (0.05);
    \draw[ptr] ($(B.south)!0.5!(B.text split)$) --++(right:10mm) |- (C.text west);
    \draw[fill] ($(C.south)!0.5!(C.text split)$) circle (0.05);
    \draw[ptr] ($(C.south)!0.5!(C.text split)$) |- (CI.text west);
    \draw[fill] ($(CI.south)!0.5!(CI.text split)$) circle (0.05);
    \draw[ptr] ($(CI.south)!0.5!(CI.text split)$) --++(right:6mm) |- (D.text west);
    \draw[fill] ($(D.south)!0.5!(D.text split)$) circle (0.05);
    \draw[ptr] ($(D.south)!0.5!(D.text split)$) --++(right:10mm)
    |- ($(last.text west) + (-0.4, 0.9)$) |- (last.text west);
  \end{tikzpicture}
  \caption{删除链表节点}
  \label{fig:list03}
\end{figure}      
\end{document}

%%% Local Variables:
%%% mode: latex
%%% TeX-master: t
%%% End:
