% 使用ctexart文档类(用XeLaTeX编译,直接支持中文)
\documentclass[12pt]{standalone}
\usepackage{tikz}
%导言区,可以在此引入必要的宏包

\begin{document} %在document环境中撰写文档
目录处理:
\tikzset{box/.style={rectangle, draw=black, fill=lightgray}} 
        \begin{tikzpicture}
          \node (begin) {};
          \node[box] (tex) [right=1 of begin] {.tex源文档};
          \node[box] (latex) [right=1 of tex] {{\LaTeX}引擎};
          \node[box] (pdfdvi)  [right=1  of latex] {PDF/DVI文件};
          \node (end)  [right=1  of pdfdvi] {};
          \node[box] (toc) [above=1.5 of latex] {.toc目录文件};
          \draw[->] (begin) -- node[above]{编写} (tex);
          \draw[->] (tex) -- node[above]{输入} (latex);
          \draw[->] (latex) -- node[above]{输出} (pdfdvi);
          \draw[->] (pdfdvi) -- node[above]{发布} (end);
          \draw[->,dashed,transform canvas={xshift = -0.3cm}] (latex) -- node[left]{前一次编译} (toc);
          \draw[->,dashed,transform canvas={xshift = 0.3cm}] (toc)-- node[right]{再一次编译} (latex);
        \end{tikzpicture}
图表引用
\tikzset{box/.style={rectangle, draw=black, fill=lightgray}} 
        \begin{tikzpicture}
          \node (begin) {};
          \node[box] (tex) [right=1 of begin] {.tex源文档};
          \node[box] (latex) [right=1 of tex] {{\LaTeX}引擎};
          \node[box] (pdfdvi)  [right=1  of latex] {PDF/DVI文件};
          \node (end)  [right=1  of pdfdvi] {};
          \node[box] (toc) [above=1.5 of latex] {.aux辅助文件};
          \draw[->] (begin) -- node[above]{编写} (tex);
          \draw[->] (tex) -- node[above]{输入} (latex);
          \draw[->] (latex) -- node[above]{输出} (pdfdvi);
          \draw[->] (pdfdvi) -- node[above]{发布} (end);
          \draw[->,dashed,transform canvas={xshift = -0.3cm}] (latex) -- node[left]{前一次编译} (toc);
          \draw[->,dashed,transform canvas={xshift = 0.3cm}] (toc)-- node[right]{再一次编译} (latex);
        \end{tikzpicture}        

参考文献        
\tikzset{box/.style={rectangle, draw=black, fill=lightgray}} 
        \begin{tikzpicture}
          \node (begin) {编写};
          \node[box] (tex) [right=1 of begin] {.tex源文档};
          \node[box] (xelatex2) [right=1 of tex] {{\XeLaTeX}};
          \node[box] (nocitepdf)  [right=1  of xelatex2] {无引用PDF};
          \node[box] (aux2)  [below=1  of xelatex2] {.aux辅助};
          \node[box] (xelatex3)  [below=1  of aux2] {\XeLaTeX};
          \node[box] (finalpdf)  [right=1  of xelatex3] {最终PDF};
          \node (end)  [right=1  of finalpdf] {发布};
          \node[box] (aux1) [above=1 of xelatex2] {.aux辅助};
          \node[box] (xelatex1) [above=2 of aux1] {\XeLaTeX};
          \node[box] (nobibpdf) [right=1 of xelatex1] {无文献PDF};
          \node[box] (bibtex) [right=2 of aux1] {bibtex};
          \node[box] (bbl) [right=1 of bibtex] {.bbl文献列表};
          \node (bibtexabove) [above=1 of bibtex] {};
          \node[box] (bib) [left=0.5 of bibtexabove] {.bib数据库};
          \node[box] (bst) [right=0.5 of bibtexabove] {.bst格式文件};
          \draw[->] (begin) -- (tex);
          \draw[->] (tex) -- node[below]{3} (xelatex2);
          \draw[->] (tex) edge [bend left] node[right]{1} (xelatex1);
          \draw[->] (tex) edge [bend right] node[right]{4} (xelatex3);
          \draw[->] (xelatex2) -- node[below]{3} (nocitepdf);
          \draw[->,dashed] (xelatex2) -- node[left]{3} (aux2);
          \draw[->,dashed] (aux2) -- node[left]{4} (xelatex3);
          \draw[->] (xelatex3) -- node[below]{4} (finalpdf);
          \draw[->] (finalpdf) -- node[below]{4} (end);
          \draw[->,dashed] (xelatex1) -- node[left]{1} (aux1);
          \draw[->] (xelatex1) -- node[above]{1} (nobibpdf);
          \draw[->,dashed] (aux1) -- node[left]{3} (xelatex2);
          \draw[->] (aux1) -- node[below]{2} (bibtex);
          \draw[->] (bib) -- node[left]{2} (bibtex);
          \draw[->] (bst) -- node[right]{2} (bibtex);
          \draw[->] (bibtex) -- node[below]{2} (bbl);
          \draw[->,dashed] (bbl) -- node[above]{3} (xelatex2);
          \draw[->,dashed] (bbl) edge [bend left] node[left]{4} (xelatex3);
        \end{tikzpicture}


\end{document}

%%% Local Variables:
%%% mode: latex
%%% TeX-master: t
%%% End:
