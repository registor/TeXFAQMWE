% 使用ctexart文档类(用XeLaTeX编译,直接支持中文)
\documentclass[12pt]{standalone}
\usepackage{ctex}
% 部分latex的Logo
\usepackage{xltxtra}

\usepackage{tikz-flowchart}
% \usepackage{tikz}
% \usetikzlibrary{positioning}
% \usetikzlibrary{calc}
% \usetikzlibrary{arrows.meta}

% 各参数默认值
\flowchartset{
  free color = green, % 自由连线颜色(默认取green)
  norm color = blue, % 常规连线颜色(默认取blue)
  cong color = red, % 关联连线颜色(默认取red)
  proc fill color = red!10, % 顺序处理框填充颜色(默认取白色)
  proc text width = 4em, % 顺序处理框宽度(默认取8em)
  chain direction = right, % 结点自动布置方向(默认取below)
  minimum node distance = 6mm, % 最小结点间距(默认取6mm)
  maximum node distance = 10mm, % 最大结点间距(默认取60mm)
  flow line width = \pgflinewidth, % 各类流程线线条宽度(默认取当前线条宽度)
  stealth length = 1.5mm, % 箭头长度(默认取1.5mm)
  stealth width = 1.0mm, % 箭头宽度(默认取1.0mm)
}

\begin{document} %在document环境中撰写文档
目录处理:

\begin{tikzpicture}[font=\scriptsize]
  \node[proc, fill=blue!30](src){.tex源文档};
  \node[proc, join](tex){{\LaTeX}引擎};
  \node[proc, join, text width = 4.5em, fill=green!30](toc){.toc(目录文件)};
  \node[proc, below=1.5 of tex](btex){{\LaTeX}引擎};
  \node[proc, join, text width = 4.5em, fill=green!30](log){.log(日志文件)};
  \node[proc, below=1.5 of btex, text width = 4.5em, fill=red!30](pdf){.pdf(排版结果)};

  \node [coord] (c1) at (src|-btex) {};
  \node [coord] (c2) at ($(toc)!0.5!(log)$) {};

  \draw [norm] (src.south) -- (c1) -- (btex);
  \draw [cong] (toc.south) -- (c2) -| (btex);

  \draw [norm] (btex.south) -- (pdf);
\end{tikzpicture}

图表引用
\begin{tikzpicture}[font=\scriptsize]
  \node[proc, fill=blue!30](src){.tex源文档};
  \node[proc, join](tex){{\LaTeX}引擎};
  \node[proc, join, text width = 4.6em, fill=green!30](toc){.aux(辅助文件)};
  \node[proc, below=1.5 of tex](btex){{\LaTeX}引擎};
  \node[proc, join, text width = 4.5em, fill=green!30](log){.log(日志文件)};
  \node[proc, below=1.5 of btex, text width = 4.5em, fill=red!30](pdf){.pdf(排版结果)};

  \node [coord] (c1) at (src|-btex) {};
  \node [coord] (c2) at ($(toc)!0.5!(log)$) {};

  \draw [norm] (src.south) -- (c1) -- (btex);
  \draw [cong] (toc.south) -- (c2) -| (btex);

  \draw [norm] (btex.south) -- (pdf);
\end{tikzpicture}


% 参考文献        
\tikzset{box/.style={rectangle, draw=black, fill=lightgray}} 
        \begin{tikzpicture}
          \node (begin) {编写};
          \node[box] (tex) [right=1 of begin] {.tex源文档};
          \node[box] (xelatex2) [right=1 of tex] {{\XeLaTeX}};
          \node[box] (nocitepdf)  [right=1  of xelatex2] {无引用PDF};
          \node[box] (aux2)  [below=1  of xelatex2] {.aux辅助};
          \node[box] (xelatex3)  [below=1  of aux2] {\XeLaTeX};
          \node[box] (finalpdf)  [right=1  of xelatex3] {最终PDF};
          \node (end)  [right=1  of finalpdf] {发布};
          \node[box] (aux1) [above=1 of xelatex2] {.aux辅助};
          \node[box] (xelatex1) [above=2 of aux1] {\XeLaTeX};
          \node[box] (nobibpdf) [right=1 of xelatex1] {无文献PDF};
          \node[box] (bibtex) [right=2 of aux1] {bibtex};
          \node[box] (bbl) [right=1 of bibtex] {.bbl文献列表};
          \node (bibtexabove) [above=1 of bibtex] {};
          \node[box] (bib) [left=0.5 of bibtexabove] {.bib数据库};
          \node[box] (bst) [right=0.5 of bibtexabove] {.bst格式文件};
          \draw[->] (begin) -- (tex);
          \draw[->] (tex) -- node[below]{3} (xelatex2);
          \draw[->] (tex) edge [bend left] node[right]{1} (xelatex1);
          \draw[->] (tex) edge [bend right] node[right]{4} (xelatex3);
          \draw[->] (xelatex2) -- node[below]{3} (nocitepdf);
          \draw[->,dashed] (xelatex2) -- node[left]{3} (aux2);
          \draw[->,dashed] (aux2) -- node[left]{4} (xelatex3);
          \draw[->] (xelatex3) -- node[below]{4} (finalpdf);
          \draw[->] (finalpdf) -- node[below]{4} (end);
          \draw[->,dashed] (xelatex1) -- node[left]{1} (aux1);
          \draw[->] (xelatex1) -- node[above]{1} (nobibpdf);
          \draw[->,dashed] (aux1) -- node[left]{3} (xelatex2);
          \draw[->] (aux1) -- node[below]{2} (bibtex);
          \draw[->] (bib) -- node[left]{2} (bibtex);
          \draw[->] (bst) -- node[right]{2} (bibtex);
          \draw[->] (bibtex) -- node[below]{2} (bbl);
          \draw[->,dashed] (bbl) -- node[above]{3} (xelatex2);
          \draw[->,dashed] (bbl) edge [bend left] node[left]{4} (xelatex3);
        \end{tikzpicture}


\end{document}

%%% Local Variables:
%%% mode: latex
%%% TeX-master: t
%%% End:
