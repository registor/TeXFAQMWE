% 使用ctexart文档类(用XeLaTeX编译,直接支持中文)
\documentclass[multi]{standalone}

%导言区,可以在此引入必要的宏包
\usepackage{ctex}
\usepackage{tikz} % Required for drawing custom shapes
% \usetikzlibrary{arrows,calc,fit,matrix,positioning,shapes,shadows,trees,mindmap,tikzmark,arrows.meta}
\usetikzlibrary{decorations.text}

\begin{document} %在document环境中撰写文档

\begin{tikzpicture}[
  text=black,
  border/.style={line width=45mm},
  every node/.style={align=center},
  pin distance=17mm,
  arc text/.style={%
      decorate,
      decoration={%
        text effects along path,
        text={#1},
        text align=center,        
        text along path,
        text effects/.cd,
        characters={anchor=mid},
      }
    }
  ]
  % 尺寸宏命令
  \def\bgr{2.5cm}
  \def\sepr{3.65cm}
  
  % 绘制  
  \foreach \txtarr/\angle/\col [remember=\angle as \last (initially 30),
  evaluate=\angle as \m using {\angle < 180 ? \angle : \last },
  evaluate=\angle as \n using { \angle < 180 ? \last : \angle } 
  ] in
  { {工具, 信息, 探究知识, 获取资源, 网络/数字化资源, 搜索引擎/工具/策略等}/90/blue!21,
    {工具, 效率, 支持构建, 提高效率, 复制/粘贴功能, 文字/图像/数据处理等}/150/cyan!33,
    {工具, 评价, 促进反思, 记录/展示, 反馈/反思, 在线批阅/电子档案等}/210/green!22,
    {工具, 认识, 高阶思维, 高阶学习, 独特形式再现知识, 专家系统/思维导图等}/270/red!18,
    {工具, 交流, {}, 协作学习, 异步/同步交流, Email/论坛/聊天等}/330/orange!24,
    {工具, 情境, 做中学, 创建情境, 情境模拟/游戏化, 案例/问题/项目等}/390/violet!19}
  {
    % 绘制背景
    \draw[\col, border] (\last:\bgr) arc(\last:\angle:\bgr);
    % 不同扇形角间的分割条    
    \draw[white, line width=2mm] (\last:1.1cm)--++(\last:\sepr);

    % 带文字装饰的圆弧路径半径
    \edef\txtr{0.01cm}
    \foreach \txt [count=\i] in \txtarr
    {
      % 根据计数奇偶性确定半径增加步长
      \pgfmathisodd{\i}
      \ifnum\pgfmathresult>0        
        \pgfmathsetmacro{\rstep}{0.3mm}
      \else
        \pgfmathsetmacro{\rstep}{0.15mm}
      \fi
      % 计算文字绘制圆弧半径
      \pgfmathparse{\txtr+\rstep}
      \xdef\txtr{\pgfmathresult}  
      % 添加路径,用文字装饰路径
      %\path[arc text/.expanded=\txt](\angle:\txtr) arc(\angle:\last:\txtr);
      \path[arc text/.expanded=\txt](\m:\txtr) arc(\m:\n:\txtr);      
    }
  }
  % 最后一个扇形分割条
  \draw[white, line width=2mm] (30:1.1cm)--++(30:\sepr);
  % 中心图
  \node[line width=1.2mm, draw, circle, minimum width=5em, white,
  fill=yellow!51,text=black] (core) {学习工具};
  % 分割圆
  \draw[white, line width = 0.5mm] (0,0) circle(2.1cm);
  \draw[white, line width = 0.5mm] (0,0) circle(3.4cm);
\end{tikzpicture}
\end{document}

%%% Local Variables:
%%% mode: latex
%%% TeX-master: t
%%% End:
