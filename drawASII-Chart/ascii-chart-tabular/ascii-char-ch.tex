\documentclass[12pt, border = 8pt, varwidth, convert]{standalone}

\usepackage{ctex}
\usepackage{collcell}

%% ======以下代码修改自:https://www.ctan.org/pkg/ascii-chart==
%% 字体设置
\font\bitfont=cmr7 \fontdimen3\bitfont=3mm
\font\codefont=cmr5
\font\namefont=cmss10 scaled 1200%ecrm1000
\font\titlefont=cmss10 scaled 1440
\font\commentfont=cmss10
%% 计数器和尺寸定义
\newdimen\thinlinewidth \thinlinewidth=.25mm
\newdimen\fatlinewidth \fatlinewidth=.5mm
\newdimen\rowheight \rowheight=1.0cm
\newdimen\colwidth  \colwidth=1.2cm
\newdimen\Colwidth \Colwidth=2\colwidth
  \advance\Colwidth by \thinlinewidth
\newdimen\topwhite \topwhite=2pt
\newdimen\botwhite \botwhite=-3pt%原值是3pt,底边空白较大?
\newdimen\leftwhite \leftwhite=2pt
\newdimen\rightwhite \rightwhite=2pt
% 用于计算ASCII码值的计数器
\newcount\rowcount \rowcount=-1 %% 要注意设置为-1
\newcount\colcount \colcount=0
\newcount\thenumber
%% 杂项设置
\def\bks{$\backslash$}
\def\thinline{\vrule width \thinlinewidth}
\def\fatline{\vrule width \fatlinewidth}
\tolerance=10000
\vbadness=10000
%% 进制转换(以下命令开始的大括号后不能为空否则会造成多余的空白)
% 计算当前行列对应的ASCII码值
\def\calcnumber{{\multiply\colcount by 16
  \advance\colcount by \rowcount
  \global\thenumber=\colcount}}
% 十进制
\def\deccode{\number\thenumber}
% 八进制
\def\octcode{{\ifnum\thenumber>63
  \advance\thenumber by -64
  \count0=\thenumber \divide\count0 by 8
  1\number\count0
  \else \count0=\thenumber \divide\count0 by 8
  \ifnum\count0>0 \number\count0 \fi\fi
  \multiply\count0 by 8
  \advance\thenumber by -\count0
  \number\thenumber}}
% 十六进制
\def\hexdigit#1{\ifcase#1 0\or 1\or 2\or 3\or 4\or 5\or 6\or 7\or
  8\or 9\or A\or B\or C\or D\or E\or F\or
  \edef\tmp{\message{illegal hex digit
  \number#1}}\tmp
  \fi}
\def\hexcode{{\count0=\thenumber \divide\count0 by 16
  \ifnum\count0>0 \hexdigit{\count0}\fi
  \multiply\count0 by 16
  \advance\thenumber by -\count0 \count0=\thenumber
  \hexdigit{\count0}}}
%% 填写表头
\def\threebit#1#2#3{\vbox to 0.8\rowheight{\bitfont
    \vskip\topwhite
    \hbox to \colwidth{\hskip\leftwhite#1\hfil
    }
    %\vskip-3.5mm
    \vfil
    \hbox to \colwidth{\hfil#2\hfil
    }
    %\vskip-3.5mm
    \vfil
    \hbox to \colwidth{\hfil#3\hskip\rightwhite
    }
    \vskip\botwhite}}
\def\comment#1{\vbox to 0.8\colwidth{\vfil%\hrule height 0mm depth .25mm
    \vfil
    \hbox to \Colwidth{\hfil#1\hfil%\commentfont
    }
    \vfil}}
\def\bithead{\vbox to 0.8\colwidth{\hsize=1.5\colwidth
    \vskip \topwhite
    \hbox to \hsize{\commentfont\hskip5.5mm BITS\hfil
    }
    %\vskip-1.0mm
    \vfil
    \hbox to \hsize{\bitfont\ b4 b3 b2 b1
    }
    \vskip\botwhite}}
%% ASCII字符排版
\def\fourbit#1\fb{\vbox to \rowheight{
    \vfil
    \hbox to 1.0\colwidth{\bitfont\hfil #1\hfil}
    \vfil%
  }%
  \global\advance\rowcount by 1%行计数器加1
  \global\colcount=0}%列计数器复位(0)
\def\asc#1\ii{\calcnumber%计算ASCII码10、16、8进制的值
  \vbox to \rowheight{\offinterlineskip
    \vskip\topwhite
    \hbox to \colwidth{\codefont
      \hskip\leftwhite
      \deccode\hfil
    }
    \vfil
    \hbox to \colwidth{%\vrule width 0cm height 10pt depth 2pt
      \namefont
      \hfil#1\hfil
    }
    \vfil
    \hbox to \colwidth{\codefont
      \hskip\leftwhite
      \hexcode\hfil\octcode
      \hskip\rightwhite
    }
    \vskip\botwhite
  }%
  \global\advance\colcount by 1% 列计数器加1
}
%% ========================================================
% 在表格列中使用带参数的命令
\newcommand*{\myfirstCol}[1]{\fourbit{#1}\fb}% Do anything you like with `#1`
\newcolumntype{F}{>{\collectcell\myfirstCol}c<{\endcollectcell}}

\newcommand*{\mymacro}[1]{\asc{#1}\ii}% Do anything you like with `#1`
\newcolumntype{C}{>{\collectcell\mymacro}c<{\endcollectcell}}

%=======================================================
\begin{document} %在document环境中撰写文档
\pagestyle{empty}
\centering
\setlength\tabcolsep{1.5pt}
%\renewcommand{\arraystretch}{0.5}
\begin{tabular}{|F|C|C|C|C|C|C|C|C|}
  \hline
  \multicolumn{1}{|r|}{\threebit{b7}{b6}{b5}}&
  \multicolumn{1}{c|}{\threebit000}&\multicolumn{1}{c|}{\threebit001}&
  \multicolumn{1}{c|}{\threebit010}&\multicolumn{1}{c|}{\threebit011}&
  \multicolumn{1}{c|}{\threebit100}&\multicolumn{1}{c|}{\threebit101}&
  \multicolumn{1}{c|}{\threebit110}&\multicolumn{1}{c|}{\threebit111}\\\cline{2-9}
  \multicolumn{1}{|c|}{\bithead}&
  \multicolumn{2}{c|}{\comment{控制字符}}&
  \multicolumn{2}{c|}{\comment{符号和数据}}&
  \multicolumn{2}{c|}{\comment{大写字母}}&
  \multicolumn{2}{c|}{\comment{小写字母}}\\\hline
  0 0 0 0&NUL&DLE&SP     &0  &@  &P       &\texttt{`} &p  \\\hline
  0 0 0 1&SOH&DC1&!      &1  &A  &Q       &a &q \\\hline
  0 0 1 0&STX&DC2&"      &2  &B  &R       &b &r \\\hline
  0 0 1 1&ETX&DC3&\#     &3  &C  &S       &c &s  \\\hline
  0 1 0 0&EOT&DC4&\$     &4  &D  &T       &d &t  \\\hline
  0 1 0 1&ENQ&NAK&\%     &5  &E  &U       &e &u  \\\hline
  0 1 1 0&ACK&SYN&\&     &6  &F  &V       &f &v  \\\hline
  0 1 1 1&BEL&ETB&'      &7  &G  &W       &g &w  \\\hline
  1 0 0 0&BS &CAN&(      &8  &H  &X       &h &x  \\\hline
  1 0 0 1&HT &EM &)      &9  &I  &Y       &i &y  \\\hline
  1 0 1 0&LF &SUB&*      &:  &J  &Z       &j &z  \\\hline
  1 0 1 1&VT &ESC&+      &;  &K  &[       &k &$\{$\\\hline
  1 1 0 0&FF &FS &,      &$<$&L  &\bks      &l &| \\\hline
  1 1 0 1&CR &GS &$-$    &=  &M  &]       &m &$\}$\\\hline
  1 1 1 0&SO &RS &.      &$>$&N  &\char94 &n &\char126\\\hline
  1 1 1 1&SI &US &/      &?  &O  &\texttt{\_}&o &DEL\\\hline
  \multicolumn{3}{r}{图例:}&
  \multicolumn{4}{c}{\hskip\fatlinewidth
      \vtop{\vskip-10pt\hbox{
          \vrule
          \vbox to 1.1\rowheight{
            %\offinterlineskip
            \hrule\vskip \topwhite
            \hbox to 1.2\colwidth{
              \codefont\hskip\leftwhite dec\hfil
            }
            \vfil
            \hbox to 1.2\colwidth{
              \namefont\hfil CHAR\hfil
            }
            \vfil
            \hbox to 1.2\colwidth{
              \codefont\hskip\leftwhite hex\hfil oct \hskip\rightwhite
            }
            \vskip-\botwhite
            \hrule}%
            \vrule
          }
        }
        \hfil
        }&
        \multicolumn{2}{c}{
          \tiny \hskip\fatlinewidth
          \vtop{\vskip-10pt\baselineskip=8.5pt
          \hbox{汉化:耿楠}
          \rlap{信息工程学院}
          \rlap{西北农林科技大学}
          \rlap{中国·陕西·杨凌}        
          }
        }%\\
\end{tabular}
\end{document}

%%% Local Variables:
%%% mode: latex
%%% TeX-master: t
%%% End:
