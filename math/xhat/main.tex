% question from https://wenda.latexstudio.net/q-1115.html
\documentclass{article}
\usepackage{amsmath}
\usepackage{bm}
\usepackage{tikz}
\usetikzlibrary{arrows.meta}
\usepackage{xparse}
\makeatletter
\newsavebox\xhat@box
\newsavebox\xhat@box@pre
\newsavebox\xhat@box@suf
\newsavebox\xhat@box@voffset
\ExplSyntaxOn
\NewDocumentCommand \xhat { > {\SplitArgument{1}{,}} O{,} O{} m }{
  \xhat@getxoffset#1
  \savebox\xhat@box@voffset{\ensuremath{#2}}
  \savebox\xhat@box{\ensuremath{#3}}
  \PackageWarning{xhat}{|#2|}{}
  \mathnormal{\mathop{
    \mathstrut
    \xhat@voffset{\xhat@box@voffset}
    \xhat@voffset{\xhat@box}
    \smash[t]{\usebox\xhat@box}}
    \limits^{
      \begin{tikzpicture}[baseline={(beg)}]
        \path 
          (.5\wd\xhat@box@pre, 0) coordinate (beg)
          (\wd\xhat@box - .5\wd\xhat@box@suf, 0) coordinate (end);
        \draw[white] (0, 0) -- +(\wd\xhat@box, 0);
        \draw
          (beg) -- node[midway, fill=black, draw=white, line~width=1pt, circle, inner~sep=1pt] {} (end);
        \path[arrows={-Latex[length=3pt]}]
          (beg) edge +(0, -5pt)
          (end) edge +(0, -5pt);
      \end{tikzpicture}
    }
  }
}
\def\xhat@getxoffset#1#2{
  \savebox\xhat@box@pre{\ensuremath{#1}}
  \savebox\xhat@box@suf{\ensuremath{#2}}
}
\def\xhat@voffset#1{
  \rule{0pt}{\dimexpr\ht#1 - 2pt\relax}
}
\ExplSyntaxOff
\makeatother
%\input code-with-output
%   code-with-output is downloadable from 
%   https://github.com/muzimuzhi/latex-examples/blob/master/utilities/code-with-output.tex
\begin{document}
%\begin{example*}{\texttt{\textbackslash xhat} usages}
  % usage
  % \xhat
  %  [, ]
  %  []
  %  {}
  
  \Large\noindent Function of \verb||:
  \begin{align*}
      abc \xhat     {abc} ffffd \xhat     {ST} \\
      abc \xhat[a,b]{abc} ffffd \xhat[S,T]{ST}
  \end{align*}
  
  \noindent Function of \verb||:
  \begin{align*}
    abc \xhat[\bm{S},\bm{T}]     {\bm{TS}} &=
      \xhat[\bm{g},\bm{g}_s]{\bm{g}^k\bm{g}^l\bm{g}_r\bm{g}_s} \\
    abc \xhat[\bm{S},\bm{T}][g^l]{\bm{TS}} &=
      \xhat[\bm{g},\bm{g}_s]{\bm{g}^k\bm{g}^l\bm{g}_r\bm{g}_s}
  \end{align*}
%\end{example*}
\end{document}

%%% Local Variables:
%%% mode: latex
%%% TeX-master: t
%%% End:
