%https://tex.stackexchange.com/questions/60812/beamer-tikz-generating-two-slides
\documentclass{beamer}
\usepackage{tikz}

\begin{document}
\begin{frame}
\frametitle{Power Utility Portfolio Choice}

\begin{itemize}
    \item Portfolio Return
        \tikz[remember picture, overlay, baseline=-.5ex]\node (n1) {};
\end{itemize}

\begin{equation*}
\begin{aligned}
& \underset{\boldsymbol w_t}{\text{maximise}}
& & \tikz[baseline,remember picture]{
        \node[fill=blue!20,anchor=base] (t1)
            {$\boldsymbol w_t \cdot \bar{\boldsymbol r}_t $};
    } + \frac{1 - \gamma}{2} 
    \tikz[baseline,remember picture]{
            \node[fill=red!20,anchor=base] (t2)
            {$ \boldsymbol w_t \cdot (\Sigma_t\boldsymbol w_t) $};
    } \\
& \text{subject to}
& & \boldsymbol w_t \cdot \boldsymbol \iota = 1\\
&&& \boldsymbol w_t \ge 0
\end{aligned}
\end{equation*}

\begin{itemize}
    \item Portfolio Variance
        \tikz[remember picture,overlay, baseline=-.5ex]\node (n2) {};
\end{itemize}

\begin{tikzpicture}[remember picture,overlay]
        \path[-stealth] (n1) edge [bend left] (t1);
        \path[-stealth] (n2) edge [out=0, in=-90] (t2);
\end{tikzpicture}

\end{frame}

\end{document}