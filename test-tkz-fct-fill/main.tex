% \documentclass[12pt, border = 8pt, varwidth, convert]{standalone}
\documentclass[margin=3pt,
   varwidth, 
  convert,
  convert={
    outext=.png,
    command=\unexpanded{
      pdftocairo -r 300 -png \infile % 将生成的pdf文件转换为png图像
    }
  }
  ]{standalone}

\usepackage{tkz-fct}

\begin{document} %在document环境中撰写文档
\begin{tikzpicture}
  % 定义坐标区域
  \tkzInit[xmin=0, xmax=6, ymin=0, ymax=6]
  % 绘制坐标轴
  \tkzDrawXY[>=stealth']
  % 绘制曲线
  \tkzFct[name path global=aa, thick,color=red, domain=0:6]{5*sin(x)}% 曲线a
  \tkzFct[name path global=bb, thick,color=blue, domain=0:3]{x**2}% 曲线b
  \tkzFct[name path global=cc, thick,color=green, domain=0.2:5]{1.0/x}% 曲线c
  \def\tkzFctgnud{0}% 曲线d
  % 求交点坐标以确定domain
  \path[name intersections={of=aa and bb}];
  \coordinate[label=above:$axb$] (CC) at (intersection-2);  
  \path[name intersections={of=bb and cc}];
  \coordinate[label=above:$bxc$] (DD) at (intersection-1);
  \newdimen\mydim
  \pgfextractx{\mydim}{(CC)}
  \pgfmathtruncatemacro{\AX}{\mydim}
  \pgfextractx{\mydim}{(DD)}
  \pgfmathtruncatemacro{\BX}{\mydim}
  \BX
  \tkzDrawAreafg[pattern=north east lines, between= a and b, color=gray!80, domain = 0:\AX]
  %\tkzDrawAreafg[pattern=dots, between= b and d, color=gray!50, domain = 0:\BX]
  \tkzDrawAreafg[pattern=dots, between= c and d, color=gray!50, domain = \BX:5]
  %   let \p1=(axb)in
  %   \pgfextra{\pgfmathsetmacro\xab{\x1/28.45}\pgfmathsetmacro\yab{\y1/28.45}};
  % \path[name intersections={of=b and c, by=bxc}]
  %   let \p1=(bxc)in
  %   \pgfextra{\pgfmathsetmacro\xbc{\x1/28.45}\pgfmathsetmacro\ybc{\y1/28.45}};  
    % 绘制填充区域
  %\coordinate[label=above:$axb$] (C) at (intersection-2);  
  
  %\tkzDrawAreafg[pattern=dots, between= b and d, color=gray!50, domain = 0:1]
  % \tkzDrawAreafg[pattern=dots, between= c and d, color=gray!50, domain = 1:5]
  % \tkzDrawAreafg[pattern=bricks, between= b and c, color=gray!50,domain = 0:3]
  % \tkzDrawAreafg[between= b and f,color=gray!80,domain = 8:18]
  % \tkzDrawAreafg[between= d and c,color=gray!50,domain = 2:20]
\end{tikzpicture}

% \begin{tikzpicture}[domain=0:4]
%     \tkzGrid
%     \tkzDrawXY[>=stealth']
%     \draw[red] plot[samples=400](\x,{5*sin(\x r)});
%     \draw[blue] plot[samples=400,domain=0:3](\x,{\x^2});
%     \begin{scope}
%         \draw plot[samples=400,domain=0:3](\x,{\x^2})--++(-3,0)--cycle;
%         \path[pattern=north east lines] (0,0)--plot[domain=0:pi](\x,{5*sin(\x r)})--cycle;
%     \end{scope}
%     \draw[green] plot[samples=400,domain=0.2:5](\x,{1/\x});
%     \begin{scope}
%         \clip (0,0)plot[domain=0:2](\x,{\x^2})--(5,2)--(5,0)--cycle;
%         \clip plot[domain=0.2:5](\x,{1/\x})--(5,0)--(0,0)--(0,5)--cycle;
%         \path[pattern=dots] (0,0)rectangle(5,3);
%     \end{scope}

%     \begin{scope}
%         \clip plot[domain=0:3](\x,{\x^2})--++(2,0)--(5,0)--cycle;
%         \clip plot[domain=0.1:5](\x,{1/\x})--(5,5)--cycle;
%         \clip plot[domain=0:pi](\x,{5*sin(\x r)})--(0,0);
%         \path[pattern=bricks] (0,0)rectangle(5,5);
%     \end{scope}
% \end{tikzpicture}
% \begin{tikzpicture}[scale=.75]
%   \tkzInit[xmax=20,ymax=12]
%   \tkzGrid[color=orange,sub](0,0)(20,12)
%   \tkzAxeXY
%   \tkzFct[samples=400,domain =0:8]{(32-4*x)**(0.5)}% a
%   \tkzFct[samples=400,domain =0:18]{(72-4*x)**(0.5)}% b
%   \tkzFct[samples=400,domain =0:20]{(112-4*x)**(0.5)}% c
%   \tkzFct[samples=400,domain =2:20]{(152-4*x)**(0.5)}% d
%   \tkzFct[samples=400,domain =12:20]{(192-4*x)**(0.5)}% e
%   \def\tkzFctgnuf{0}% f
%   \def\tkzFctgnug{12}% g
%   % \tkzDrawAreafg[between= b and a,color=gray!80,domain = 0:8]
%   \tkzDrawAreafg[between= b and f,color=gray!80,domain = 8:18]
%   % \tkzDrawAreafg[between= d and c,color=gray!50,domain = 2:20]
%   \tkzDrawAreafg[between= g and c,color=gray!50,domain = 0:2]
%   % \tkzDrawAreafg[between= g and e,color=gray!20,domain =12:20]
% \end{tikzpicture}%
\end{document}

%%% Local Variables:
%%% mode: latex
%%% TeX-master: t
%%% End:
