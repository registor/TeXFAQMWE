\documentclass[tikz,margin=10pt]{standalone}

\usepackage{ctex}
\usepackage{makecell} % 单元格宏包

\usetikzlibrary{arrows.meta}
\usetikzlibrary{decorations.text}

% 绘制五行图的颜色定义
\definecolor{arylideyellow}{rgb}{0.91, 0.84, 0.42}
\definecolor{forestgreen}{rgb}{0.13, 0.55, 0.13}
\definecolor{mordantred19}{rgb}{0.68, 0.05, 0.0}

\tikzset{
  arcnode/.style 2 args={
    decoration={
      text along path,
      text align=center,
      raise=#1,
      text={|\small|{#2}}},
      postaction={decorate},
    },
  % 节点文本框样式
  MN/.style args = {#1/#2}{
    draw=#1,% line color
    top color=#2!10,
    bottom color=#2!80,
    circle,
    % rounded corners,
    thick,
    % text width=20mm,
    % minimum height=13mm,
    inner sep=0.5mm,
    align=flush center
  },
  % 节点说明文本框样式
  AN/.style args = {#1/#2}{
    draw=#1,% line color
    top color=#2!10,
    bottom color=#2!80,          
    rounded corners,
    thick,
    % text width=20mm,
    % minimum height=13mm,
    inner sep=1.0mm,
    align=flush center
  },
  % 线条样式
  line/.style = {
    line width=0.3mm,
    draw=#1,%line color
    -{Stealth[length=1.5mm,width=1.0mm,fill=#1]},
    shorten >=1mm,
    shorten <=1mm
  },
  % 阴影
  ds/.style = {drop shadow}
}

\begin{document} %在document环境中撰写文档
\begin{tikzpicture}%[]
  % 定义绘图尺寸宏
  \def\n{5}
  \def\radius{1.6cm}
  \def\margin{8.5} % 角度边距,依赖于radius

  % 布置节点
  \foreach \clr/\v [count=\rang from 0] in {orange/木,green/火,magenta/土,blue/金,arylideyellow/水}
  {
    \node[MN=gray/\clr!40!white] at ({360/\n * -\rang + 90}:\radius) {\v};        
  }

  % 相生标签
  \foreach \gen [count=\rang from 0] in {木生火, 火生土, 土生金, 金生水, 水生木}
  {
    \draw[line=forestgreen,
     arcnode={3pt}{\textcolor{forestgreen}{\tiny\gen}}] ({360/\n * -\rang +
      90 - \margin}:\radius) arc ({360/\n * -\rang + 90 -
      \margin}:{360/\n * -(\rang + 1) + 90 + \margin}:\radius);           
  }

  \foreach \lab [count=\rang from 0] in {木克土, 火克金, 土克水, 金克木, 水克火}
  {
    \draw[line=red,
    arcnode={2pt}{\textcolor{mordantred19}{\tiny\lab}}] ({360/\n *
      -\rang + 90 - 4}:{\radius - 6.0}) -- ({360/\n * -(\rang + 2)
      + 90 + 4}:{\radius - 6.0});         
  }
\end{tikzpicture}
\end{document}

%%% Local Variables:
%%% mode: latex
%%% TeX-master: t
%%% End:
