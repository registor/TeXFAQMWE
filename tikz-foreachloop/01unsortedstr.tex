\documentclass[tikz,margin=5pt]{standalone}

% 支持中文
\usepackage{ctex}
% TikZ宏包扩展
\usetikzlibrary{positioning}
\usetikzlibrary{arrows.meta}

% 语法高亮显示排版代码宏包
\usepackage{minted}

% 自定义代码排版命令与环境
% =============================================================================
\usemintedstyle{default}  %codeblocks模式
% 共用设置
\setminted{fontsize=\tiny, breaklines=true, breakautoindent=false}
% 不同字号的行内代码命令
\newmintinline[cinttscr]{c}{fontsize=\scriptsize, escapeinside=||, autogobble}
\newmintinline[cintttny]{c}{fontsize=\tiny, escapeinside=||, autogobble}
% 代码排版环境
\newminted{c}{frame=lines, autogobble}
% =============================================================================

\begin{document} %在document环境中撰写文档
\begin{tikzpicture}[font=\scriptsize,
  x=4.5mm, y=5.0mm,
  cell/.style={draw,
    minimum size=4.5mm,
    inner sep=0pt,
  },
  cellp/.style={draw,
    minimum size=5.0mm,
    inner sep=0pt,
  },
  >=stealth,
  ]

  % 定义数组
  \def\planets{{M,e,r,c,u,r,y}, {V,e,n,u,s}, {E,a,r,t,h}, {M,a,r,s},
    {J,u,p,i,t,e,r}, {S,a,t,u,r,n}, {U,r,a,n,u,s}, {N,e,p,t,u,n,e},
    {P,l,u,t,o}}%
    
  \foreach \l [count=\i] in \planets % \l宏分别取值:{M,e,r,c,u,r,y}、{V,e,n,u,s}等
  {
    \pgfmathsetmacro{\idx}{int(\i-1)} % 计数是从1开始的,减1变为从0开始
    \node[cellp,label=left:\scriptsize \cinttscr{planets[}\idx \cinttscr{]}] (p\i) at (0, 1-\i) {};% 绘制竖向指针数组

    \edef\j{1} % 记录最后一次的\cnt
    \foreach \m [count=\cnt] in \l % \m是{M,e,r,c,u,r,y}中的每一个字母 cnt是计数值
    {
      \node[cell](str\i\cnt) at (\cnt + 1, -\i + 1) {\centering \m};% 绘制横向字符数组
      \xdef\j{\cnt} % 将计数值赋给\j
    }
    \pgfmathsetmacro{\inull}{int(\j+1)} % 计算空字符框编号
    \node[cell, right=-0.11mm of str\i\j](str\i\inull) {\centering \cintttny{'\0'}};% 绘制结束空字符
    \draw[fill] (p\i.center) circle (0.05); % 绘制指针数组元素框中心圆点
    \draw[-{Stealth[scale=1.0]}, magenta, thick] (p\i.center) -- (str\i1);% 绘制指针数组元素框中心到字符数组的指向线
  }
      
  % 绘制代码
  \node[scale = 1.0, align = left, text width=0.60\textwidth, above= 0.5 of str12](code1) 
  {
    \begin{ccode}
      char *planets[] = {"Mercury", "Venus", "Earth", "Mars", "Jupiter", 
                          "Saturn", "Uranus", "Neptune", "Pluto"};
    \end{ccode}
  };
\end{tikzpicture}  
\end{document}

%%% Local Variables:
%%% mode: latex
%%% TeX-master: t
%%% End:
