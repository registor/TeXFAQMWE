% 使用ctexart文档类(用XeLaTeX编译,直接支持中文)
\documentclass{article}

% 导言区,可以在此引入必要的宏包
\usepackage{ctex}
\usepackage[edges]{forest}

%% 设置绘制目录结构的宏参数
\definecolor{folderbg}{RGB}{124,166,198}
\definecolor{folderborder}{RGB}{110,144,169}
\newlength\Size
\setlength\Size{4pt}
\tikzset{%
  folder/.pic={%
    \filldraw [draw=folderborder, top color=folderbg!50, bottom
    color=folderbg] (-1.05*\Size,0.2\Size+5pt) rectangle
    ++(.75*\Size,-0.2\Size-5pt);
    
    \filldraw [draw=folderborder, top color=folderbg!50, bottom
    color=folderbg] (-1.15*\Size,-\Size) rectangle (1.15*\Size,\Size);    
  },
  file/.pic={%
    \filldraw [draw=folderborder, top color=folderbg!5, bottom
    color=folderbg!10] (-\Size,.4*\Size+5pt) coordinate (a) |-
    (\Size,-1.2*\Size) coordinate (b) -- ++(0,1.6*\Size) coordinate
    (c) -- ++(-5pt,5pt) coordinate (d) -- cycle (d) |- (c) ;    
  },
}

\forestset{%
  declare autowrapped toks={pic me}{},
  declare boolean register={pic root},
  pic root=0,
  pic dir tree/.style={%
    for tree={%
      folder,
      %font=\ttfamily,
      grow'=0,
      s sep=1.0pt,
      font=\small \sffamily,
      %fit=band,
      %ysep = 1.0pt,
      inner ysep = 2.6pt,
    },
    before typesetting nodes={%
      for tree={%
        edge label+/.option={pic me},
      },
      if pic root={
        tikz+={
          \pic at ([xshift=\Size].west) {folder};
        },
        align={l}
      }{},
    },
  },
  pic me set/.code n args=2{%
    \forestset{%
      #1/.style={%
        inner xsep=2\Size,
        pic me={pic {#2}},
      }
    }
  },
  pic me set={directory}{folder},
  pic me set={file}{file},  
}
%% ==================================================

\begin{document} %在document环境中撰写文档
%\centering
\begin{forest}
  pic dir tree,
  pic root,
  for tree={% folder icons by default; override using file for file icons
            directory,
           },
  [jobname\%【工作根目录】%, 
    [bibs\%【参考文献数据库目录,根据需要,可以有多个数据库】%
      [sample.bib\%【参考文献数据库】, file%
      ]
      [$\cdots$,file]
    ]
    [codes\%【源代码目录,可根据需要增减】
      [chap01
        [01helloworld.c\%【输入输出C语言代码】,file]
        [$\cdots$,file]
      ]
      [chap02
        [01bubblesort.c\%【冒泡排序C语言代码】,file]
        [$\cdots$,file]
      ]  
      [$\cdots$,file]
    ]  
    [contents\%【各章节\LaTeX 源文件,可根据需要增减】        
      [chap00-abs\%【摘要】.tex, file
      ]
      [chap01-intro.tex\%【第1章】, file
      ]
      [$\cdots$, file
      ]
      [chap10-conclusion.tex\%【第10章】, file
      ]
      [app01-denotation.tex\%【主要符号对照表】, file
      ]
      [app02-ack.tex\%【致谢】, file
      ]
      [app03-resume.tex\%【个人简历】, file
      ]
    ]      
    [datasets\%【数据文件,可根据需要增减】        
      [tab01-test01.csv\%【XXXX测试数据】, file
      ]
      [$\cdots$, file
      ]
    ]  
    [figs\%【插图目录,可根据需要增减】
      [chap01
        [fig01-01.tex\%【TiKZ代码绘图】,file]
        [fig01-02.png\%【屏幕截图】,file]
        [fig01-03.jpg\%【现场/实物照片】,file]
        [fig01-04.pdf\%【Matlab制图】,file]
        [$\cdots$,file]
      ]
      [chap02
        [$\cdots$,file]
      ]
      [$\cdots$,file]
    ]
    [logo\%【学校LOGO图标】      
      [XXXX.png\%【截图(不推荐)】, file
      ]
      [XXXX.pdf\%【矢量图标(强烈推荐)】, file
      ]
      [$\cdots$,file]
    ]
    [settings\%【自定义命令、环境、引入宏包等\LaTeX{}源文件,可根据需要调整】        
      [format.tex\%【自定义命令、环境、参数设置等】, file
      ]
      [packages.tex\%【宏包引入】, file
      ]
    ]
    [main.tex\%【主控文件,必须存在,且置于根目录】, file
    ]
    [Makefile\%【make命令需要的文件,若不执行make命令,可以不需要】, file
    ]
    [$\cdots$\%【其它必须置于根目录下的文件,可以不需要】,file]
  ]
\end{forest}

\end{document}

%%% Local Variables:
%%% mode: latex
%%% TeX-master: t
%%% End:
