\documentclass[margin=3pt,
  convert,
  convert={
    outext=.png,
    command=\unexpanded{
      pdftocairo -r 300 -png \infile % 将生成的pdf文件转换为png图像
    }
  }
  ]{standalone}

\usepackage{tkz-euclide}

\begin{document}
        
\begin{tikzpicture}[scale=1.00]
  \tkzInit
  \tkzDefPoint(0,0){A}% 定义A点
  \tkzDefShiftPoint[A](30:1){B}% 采用坐标平移定义B点
  \tkzDefShiftPoint[B](60:1.5){C}% 采用坐标平移定义C点
  % 用A为圆心AB为半径的圆与用B为圆心BA为半径的圆的交点
  % 定义等腰三角形的顶点D  
  \tkzInterCC(A,B)(B,A) \tkzGetPoints{D}{D'} 
  \tkzDrawPolygon(A,B,D)% 绘制等腰三角形
  \tkzMarkSegments[mark=|](A,D B,D)% 绘制等腰三角形标记
  % 绘制A、B、C、D点标签
  \tkzLabelPoints[](A)
  \tkzLabelPoint[below](B){$B$}
  \tkzLabelPoint[above right](C){$C$}
  \tkzLabelPoint[above](D){$D$}
  % 采用坐标平移及比例法定义E、F点
  \begin{scope}[shift=(D)]
    \tkzDefBarycentricPoint(D=-2.5,A=3.5)
    \tkzGetPoint{E}
    \tkzDefBarycentricPoint(D=-2.5,B=3.5)
    \tkzGetPoint{F}    
  \end{scope}
  % 绘制辅助线段及E、F点标记
  \tkzDrawSegment[dashed](A,E)
  \tkzLabelPoint[left](E){$E$}
  \tkzDrawSegment[dashed](B,F)
  \tkzLabelPoint[below left](F){$F$}
  % 绘制圆心为B,半径为BC的圆
  \tkzDrawCircle[blue!50,dashed](B,C)
  % 求圆与直线DB的交点G
  \tkzInterLC(B,D)(B,C) \tkzGetPoints{G}{G'}
  % 绘制G、H标签
  \tkzLabelPoint[below](G){$G$}
  \tkzLabelPoint[above,shift={(0.5,0.55)}](G'){$H$}
  % 绘制圆心为D,半径为DG的圆
  \tkzDrawCircle[blue!20,dashed](D,G)
  % 求圆与直线DA的交点L
  \tkzInterLC(A,D)(D,G) \tkzGetPoints{L}{L'}
  % 绘制L、K标签
  \tkzLabelPoint[above,shift={(-1.8,-0.7)}](L'){$K$}
  \tkzLabelPoint[below left](L){$L$}
  % 绘制求得的线段AL、BG、BC
  \tkzDrawSegments[red](A,L B,G B,C)
  % 绘制线段相等标记
  \tkzMarkSegments[mark=||](A,L B,G B,C)  
\end{tikzpicture}
\end{document}

%%% Local Variables:
%%% mode: latex
%%% TeX-master: t
%%% End:
