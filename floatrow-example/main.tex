% 使用ctexart文档类(用XeLaTeX编译,直接支持中文)
\documentclass{ctexart}

% 导言区,可以在此引入必要的宏包
% 插图宏包
\usepackage{graphicx}
% 灵活排版浮动体
\usepackage{floatrow}
% 子标题宏包
\usepackage{subcaption}


\begin{document} %在document环境中撰写文档
在浮动体中并排插入两个元素,按照题注规律,大致有以下一些组
合:

\begin{itemize}
\item 无题注
\item 共享题注无子题注
\item 各自独立题注
\item 共享题注带子题注
\end{itemize}

\section{共享题注无子题注}
\begin{figure}[!htp]
  \begin{floatrow}
    \ffigbox[\FBwidth]{
      \includegraphics[width=0.45\textwidth]{example-image-a}
      \includegraphics[width=0.45\textwidth]{example-image-b}
    }{\caption{共享题注}\label{sharefig:a}}
  \end{floatrow}
\end{figure}

\section{各自独立题注}
\begin{figure}[!htp]
  \begin{floatrow}
    \ffigbox[\FBwidth]{
      \includegraphics[width=0.45\textwidth]{example-image-a}
    }{\caption{独立题注}\label{sepfig:a}}
    \ffigbox[\FBwidth]{
      \includegraphics[width=0.45\textwidth]{example-image-b}
    }{\caption{独立题注}\label{sepfig:b}}
  \end{floatrow}
\end{figure}

\section{共享题注带子题注}
\begin{figure}[!htp]
  \begin{floatrow}
    \ffigbox[\textwidth]{
      \begin{subfloatrow}[2]
        \ffigbox[\FBwidth]{
          \includegraphics[width=0.45\textwidth]{example-image-a}
        }{\caption{子题注}\label{subfig:a}}
        \ffigbox[\FBwidth]{
          \includegraphics[width=0.45\textwidth]{example-image-b}
        }{\caption{子题注}\label{subfig:b}}
      \end{subfloatrow}
    }{\caption{共享题注带子题注}\label{fig:sub}}
  \end{floatrow}
\end{figure}

\begin{figure}[!htp]
  \begin{floatrow}
    \ffigbox[\FBwidth]
    {
      \begin{subfloatrow}[2]
        \ffigbox[\FBwidth]{
          \includegraphics[height=0.16\textheight]{example-image-a}
        }{\caption{子题注1}\label{sepsubfig:a}}
        \ffigbox{
          \includegraphics[height=0.16\textheight]{example-image-b}
        }{\caption{子题注2}\label{sepsubfig:b}}
      \end{subfloatrow}
    }{\caption{独立题注1}\label{sepsubfig:ab}}
    \ffigbox{
      \includegraphics[height=0.16\textheight]{example-image-c}\\[5.5ex]
    }{\caption{独立题注2}\label{sepsubfig:c}}
  \end{floatrow}
\end{figure}


\begin{figure}[!htp]
  \ffigbox[\textwidth]%
  {%
    \begin{subfloatrow}[1]%\useFCwidth
      \ffigbox[\FBwidth]{
        \includegraphics[width=0.4\textwidth]{example-image-a}
      }{\caption{子题注1}\label{trifig:a}}
    \end{subfloatrow}    
    \begin{subfloatrow}[2]%\useFCwidth      
      \ffigbox[\FBwidth]{
        \includegraphics[width=0.4\textwidth]{example-image-b}
      }{\caption{子题注2}\label{trifig:b}}
      \ffigbox[\FBwidth]{
        \includegraphics[width=0.4\textwidth]{example-image-c}
      }{\caption{子题注3}\label{trifig:c}}
    \end{subfloatrow}
  }{\caption{三个子图}\label{trifig}}
\end{figure}

\begin{figure}[!htp]
  \ffigbox[\textwidth]%
  {%
    \begin{subfloatrow}[2]%\useFCwidth
      \ffigbox[\FBwidth]{
        \includegraphics[width=0.4\textwidth]{example-image-a}
      }{\caption{子题注1}\label{trifig:a}}
      \ffigbox[\FBwidth]{
        \includegraphics[width=0.4\textwidth]{example-image-b}
      }{\caption{子题注2}\label{trifig:b}}
    \end{subfloatrow}    
    \begin{subfloatrow}[2]%\useFCwidth      
      \ffigbox[\FBwidth]{
        \includegraphics[width=0.4\textwidth]{example-image-c}
      }{\caption{子题注3}\label{trifig:c}}
      \ffigbox[\FBwidth]{
        \includegraphics[width=0.4\textwidth]{example-image}
      }{\caption{子题注4}\label{trifig:d}}
    \end{subfloatrow}
  }{\caption{四个子图}\label{trifig}}
\end{figure}



\end{document}

%%% Local Variables:
%%% mode: latex
%%% TeX-master: t
%%% End:
