\documentclass{ctexart}

\usepackage{expl3}
\usepackage{etoolbox}

% 声中文明强调字体及强调方式
% 用\bfseries表示强调
\makeatletter
\newcommand*\emphfont{\normalfont\bfseries}
\DeclareTextFontCommand\@textemph{\emphfont}
\newcommand\textem[1]{%
  \ifdefstrequal{\f@series}{\bfdefault}
    {\@textemph{\CJKunderline{#1}}}% 用下划线表示强调中的强调
    {\@textemph{#1}}%
}
\makeatother

% 重新定义\emph强调命令
% 参考:https://tex.stackexchange.com/questions/13048/upright-parentheses-in-italic-text?rq=1
\ExplSyntaxOn
\cs_new_eq:Nc \emph_old:n { emph~ } % 备份`\emph`命令
\cs_new_protected:Npn \emph_braces:n #1 % 排版括号为直立符号
{ \mode_if_math:TF {#1} { \textup{#1} } }

\cs_new:Npn \emph_new:n #1 {
  \tl_set:Nn \l_emph_tl {\textem{#1}}% 中文强调时用黑体
  \tl_replace_all:Nnn \l_emph_tl {(}{\emph_braces:n{(}}% 替换(为直立
  \tl_replace_all:Nnn \l_emph_tl {)}{\emph_braces:n{)}}% 替换)为直立
  \tl_replace_all:Nnn \l_emph_tl {[}{\emph_braces:n{[}}% 替换[为直立
  \tl_replace_all:Nnn \l_emph_tl {]}{\emph_braces:n{]}}% 替换]为直立
  \exp_args:NV \emph_old:n \l_emph_tl
}
\RenewDocumentCommand {\emph} {sm} {% \emph*是原命令,\emph为重新定义的命令
  \IfBooleanTF {#1} {\emph_old:n {#2}} {\emph_new:n {#2}}
}
\ExplSyntaxOff

\begin{document}
采用\verb|\bfseries|粗体表示强调,而不是采用\emph*{italic}字形表示强调。


用下划线表示强调中的强调。

在强调内容中,如果存在\verb|(、)、[、]|,则不使用\emph*{italic}体。

如:

\emph{欢迎}来到\LaTeX 世界!

\emph{Welcome} to \LaTeX World!

\emph{\emph{欢迎}来到\LaTeX 世界!}

\emph{\emph{Welcome} to \LaTeX (World)!}

如果需要使用原始强调命令,则使用\verb|\emph*|命令,如

\emph*{欢迎}来到\LaTeX 世界!

\emph*{Welcome} to \LaTeX World!

\emph*{\emph*{欢迎}来到\LaTeX 世界!}

\emph*{\emph*{Welcome} to \LaTeX (World)!}

对数学公式,则没有影响,如:

some math \emph{\((x+y)^2\) and text (again)}.

\end{document}
%%% Local Variables:
%%% mode: latex
%%% TeX-master: t
%%% End:
