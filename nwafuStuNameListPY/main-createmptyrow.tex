% 使用ctexart文档类(用XeLaTeX编译,直接支持中文)
\documentclass{ctexart}

\usepackage{geometry}
\geometry{% 本科生页边距
  paper=a4paper,
  top=2.5cm,
  bottom=2.1cm,
  left=2.5cm,
  right=2.5cm,
  headheight=0.8cm,
  headsep=0.7cm,
  footskip=1.4cm,
} 

\usepackage{pgfplotstable}
\usepackage{multirow}
% recommended:
%\usepackage{booktabs}
%\usepackage{array}
%\usepackage{colortbl}

\usepackage{xpinyin}
%\xpinyinsetup{vsep=0.9em}

\pgfplotstableset{
  col sep=comma,
  header=false,
  font=\small,
}

%\pgfplotstableread{namelist-head.csv}{\loadedtable}
\pgfplotstableread{namelist-nohead.csv}{\loadedtable}
% 创建空白列
\pgfplotstablecreatecol{A}\loadedtable
\pgfplotstablecreatecol{B}\loadedtable
\pgfplotstablecreatecol{C}\loadedtable
\pgfplotstablecreatecol{D}\loadedtable
\pgfplotstablecreatecol{E}\loadedtable
\pgfplotstablecreatecol{F}\loadedtable
\pgfplotstablecreatecol{G}\loadedtable
\pgfplotstablecreatecol{H}\loadedtable
\pgfplotstablecreatecol{I}\loadedtable
\pgfplotstablecreatecol{J}\loadedtable
\pgfplotstablecreatecol{K}\loadedtable
\pgfplotstablecreatecol{L}\loadedtable
\pgfplotstablecreatecol{M}\loadedtable


\begin{document} %在document环境中撰写文档
\pagestyle{empty}
\newcolumntype{C}{>{\centering\arraybackslash}p{0.5em}}% a centered fixed-width-column
\def\arraystretch{1.2}
\pgfplotstabletypeset[
  % columns={stno,name,1,2},
  column type/.add={|}{},% results in ’|c’
  %column type/.add={|@{\extracolsep{0.5cm}}}{@{\extracolsep{0.5cm}}},% results in ’|c’
  every head row/.style={
    % patches the first row:
    before row={
      \multicolumn{16}{c}{\bf\large 西北农林科技大学2019秋季学期学生课程考勤表}\\
      \multicolumn{16}{l}{课程名称:C语言程序设计\hfill 课程代码:1091102\hfill 课序号:2\hfill   任课教师:耿楠}\\
      \hline
      \multirow{2}{*}{学号}&\multirow{2}{*}{姓名}&\multirow{2}{*}{班级}&\multicolumn{13}{c|}{平时考勤}\\
      \cline{4-16}
    }, 
  },
  after row={\hline},
  every last row/.style={
    after row={
      \hline
      \multicolumn{16}{|l|}{备注:出勤:$\surd$\ 缺勤:$\times$\ 迟到:$\bigtriangleup$\ 早退:$\bigcirc$\ 事假:$\bigtriangledown$\ 病假:$\oplus$}\\
      \hline
    },
  },
  columns/0/.style={string type,
    column name={},
    %column type=|c,
  },
  columns/1/.style={string type,
    column name={},
    postproc cell content/.style={@cell content=\xpinyin*{##1}},
  },
  columns/2/.style={string type,
    column name={},
  },
  columns/A/.style={
    column name={1},
    column type=|C,
  },
  columns/A/.style={
    column name={1},
    column type=|C,
  },
  columns/B/.style={
    column name={2},
    column type=|C,
  },
  columns/C/.style={
    column name={3},
    column type=|C,
  },
  columns/D/.style={
    column name={4},
    column type=|C,
  },
  columns/E/.style={
    column name={5},
    column type=|C,
  },
  columns/F/.style={
    column name={6},
    column type=|C,
  },
  columns/G/.style={
    column name={7},
    column type=|C,
  },
  columns/H/.style={
    column name={8},
    column type=|C,
  },
  columns/I/.style={
    column name={9},
    column type=|C,
  },
  columns/J/.style={
    column name={10},
    column type=|C,
  },
  columns/K/.style={
    column name={11},
    column type=|C,
  },
  columns/L/.style={
    column name={12},
    column type=|C,
  },
  columns/M/.style={
    column name={备注},
    column type=|c|,
  },
]{\loadedtable}

\end{document}

%%% Local Variables:
%%% mode: latex
%%% TeX-master: t
%%% End:
