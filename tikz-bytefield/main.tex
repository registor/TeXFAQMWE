\documentclass[12pt, border = 8pt, varwidth, convert]{standalone}

% 中文支持
\usepackage{ctex}

% TikZ
\usepackage{tikz}
\usetikzlibrary{calc}% 计算库
\usetikzlibrary{arrows.meta}% 箭头库

% 提供\IfStrEq
\usepackage{xstring}

% 字节图绘制
\usepackage{bytefield}

% =============绘制内存图中一个内存块的自定义命令======================
% 需要bytefield宏包支持,其开始地址在底部,结束地址在顶部
% 语法:
% \memsec{结束地址}{开始地址}{以行为单位的高度}{盒子中的文字}
\newcommand*\istempaddr[1]{\IfStrEq{}{#1}{}{0X#1}}
\newcommand{\memsec}[4]{%
  % 定义memsection的高度
  \bytefieldsetup{bitheight=#3\baselineskip}%
  \bitbox[]{8}{%
    %\vspace{-#3pt}
    \footnotesize \texttt{\istempaddr{#1}}%0X#1}% 结束地址
    \\
    % 留出空白
    \vspace{#3\baselineskip}
    \vspace{-1.7\baselineskip}
    \vspace{-#3pt}
    \texttt{\istempaddr{#2}}%0X#2}% 开始地址
  }%
  \bitbox{8}{\footnotesize #4}% 盒子里的内容
}

% 自定义垂直省略号
\makeatletter
\DeclareRobustCommand{\rvdots}{%
  \vbox{
    \baselineskip4\p@\lineskiplimit\z@
    \kern-\p@
    \hbox{.}\hbox{.}\hbox{.}
  }}
\makeatother

\begin{document} %在document环境中撰写文档
% 基本内在结构图
\begin{bytefield}{8}
  \memsec{}{}{1}{\rvdots}\\
  \memsec{FFFF E31C}{}{1}{$\cdots$}\\
  \memsec{FFFF E31B}{}{1}{$\cdots$}\\
  \memsec{FFFF E31A}{}{1}{$\cdots$}\\
  \memsec{FFFF E319}{}{1}{$\cdots$}\\
  \memsec{FFFF E318}{}{1}{$\cdots$}\\
  \memsec{FFFF E317}{}{1}{$\cdots$}\\
  \memsec{FFFF E316}{}{1}{$\cdots$}\\
  \memsec{FFFF E315}{}{1}{$\cdots$}\\
  \memsec{FFFF E314}{}{1}{$\cdots$}\\
  \memsec{FFFF E313}{}{1}{$\cdots$}\\
  \memsec{FFFF E312}{}{1}{$\cdots$}\\
  \memsec{FFFF E311}{}{1}{$\cdots$}\\
  \memsec{FFFF E310}{}{1}{$\cdots$}\\
  \memsec{}{}{1}{\rvdots}
\end{bytefield}
% 添加标记
\begin{bytefield}{24}
  \memsec{}{}{1}{\rvdots}\\
  \begin{rightwordgroup}{dfValue}
    \memsec{FFFF E31F}{FFFF E318}{8}{double}
  \end{rightwordgroup}\\
  \begin{rightwordgroup}{iValue}
    \memsec{FFFF E317}{FFFF E314}{4}{int}
  \end{rightwordgroup}\\
  \begin{rightwordgroup}{ch}
    \memsec{FFFF E313}{}{1}{char}
  \end{rightwordgroup}\\
  \memsec{}{}{1}{\rvdots}
\end{bytefield}

\vspace{4ex}
% 结合TikZ绘制
\centering
\def\scalerate{1.0} %缩放因子
% 坐标计算中需要的宏定义
% 内存地址高度和宽度
\def\addrh{0.45} %高度宏
\def\addrw{2.25} %宽度宏
% 内存地址x, y偏移
\def\addrox{-3.50} %x
\def\addroy{-2.75} %y
% 内存单元高度和宽度
\def\memh{0.475} %高度宏
\def\memw{2.66} %宽度宏
% 内存单元x, y偏移
\def\memox{-1.06} %x
\def\memoy{-0.95} %y
\begin{tikzpicture}[
  thick,
  scale=\scalerate,
  every node/.style={scale=\scalerate}
  ]
  \node (fig) at (0, 0) {
    \begin{bytefield}{24}
      \begin{rightwordgroup}{piValue}
        \memsec{FFFF E317}{FFFF E310}{8}{0XFFFF E30C}
      \end{rightwordgroup}\\
      \begin{rightwordgroup}{iValue}
        \memsec{FFFF E30F}{FFFF E30C}{4}{0X00 00 00 61}
      \end{rightwordgroup}
    \end{bytefield}
  };
  % 计算绘图坐标
  \path let \p1=(fig) in
  coordinate (org) at (\x1, \y1)
  coordinate (ovBL1) at ($(org) + (\addrox, \addroy)$)
  coordinate (ovUR1) at ($(ovBL1) + (\addrw, \addrh)$)
  coordinate (ovCD1) at ($(ovBL1)!0.5!(ovUR1)$)
  coordinate (ovML1) at (ovBL1|-ovCD1)            
  coordinate (ovMLO1) at ($(ovML1) + (-0.20, 0.00)$)
            
  coordinate (ovBL2) at ($(org) + (\memox, \memoy)$)%-0.9, -0.82)$)
  coordinate (ovUR2) at ($(ovBL2) + (\memw, 8 * \memh)$)
  coordinate (ovCD2) at ($(ovBL2)!0.5!(ovUR2)$)
  coordinate (ovML2) at (ovBL2|-ovCD2)
  coordinate (ovMLO2) at (ovMLO1|-ovML2)
  coordinate (ovVN1) at ($(ovML2) + (\memw + 2.15, 0.00)$)
  coordinate (ovVN2) at ($(ovVN1) + (-0.30, -2.80)$)
  coordinate (ovPT) at ($(ovVN1) + (0.00, 1.50)$) ;

  \node[draw] (dat) at (ovPT) {97};

  \draw[red,thick,fill=green!35, fill opacity=0.3](ovBL1) rectangle (ovUR1);      
  \draw[blue,thick,fill=red!35, fill opacity=0.3](ovBL2) rectangle (ovUR2);

  \draw[-{Stealth[scale=1.0]}, red, thick] (ovML1) --
  (ovMLO1) -- node[left,align=right,yshift=0.0cm,black]{取\\地\\址}(ovMLO2)--(ovML2);
      
  \draw[-{Stealth[scale=1.0]}, blue, thick] (ovVN1) to [out = -45, in = 45]
  node[right,align=left,yshift=0.0cm,black]{指\\向}(ovVN2);

  \draw[-{Stealth[scale=1.0]}, blue, thick] (dat) to [out = -90, in = 90]
  node[right,align=left,xshift=0.3cm,yshift=0.2cm,black]{赋\\值}(ovVN1);
\end{tikzpicture}

\end{document}

%%% Local Variables:
%%% mode: latex
%%% TeX-master: t
%%% End:
