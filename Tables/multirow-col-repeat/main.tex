\documentclass{ctexart}
% 合并表格的行
\usepackage{multirow}

% 使用<{\centering}需引入宏包\usepackage{makecell}
\usepackage{makecell}

% 带圈数字
\usepackage{tikz}
\newcommand*{\circled}[1]{\lower.7ex\hbox{\tikz\draw (0pt, 0pt)%
            circle (.5em) node {\makebox[1em][c]{\small #1}};}}

\begin{document}
  \begin{table}[htp]
    \begin{tabular}{*{6}{|p{1cm}<{\centering}}|}%重复设置列格式,类似于{|c|c|c|c|c|c|}
      \hline
      \multicolumn{3}{|c|}{到达地点\circled{3}的时间} & \multicolumn{3}{c|}{到达地点\circled{6}的顺序} \\ \hline
      C & B & A & C & B & A\\ \hline
      \multirow{9}{*}{13:00} & \multirow{5}{*}{13:30} & \multirow{3}{*}{14:00} & 3 & 2 & 1  \\ \cline{4-6}
      &  &  & 3 & 1 & 2  \\ \cline{4-6}
      &  &  & 2 & 1 & 3  \\ \cline{3-6}
      &  & \multirow{2}{*}{14:30} & 3 & 1 & 2  \\ \cline{4-6}
      &  &  & 2 & 1 & 3  \\ \cline{2-6}
      & \multirow{4}{*}{14:00}  &  \multirow{2}{*}{13:30} & 2 & 3 & 1  \\ \cline{4-6}
      &  &  & 1 & 3 & 2  \\ \cline{3-6}
      &  &  \multirow{2}{*}{14:30} & 1 & 3 & 2  \\ \cline{4-6}
      &  &  & 1 & 2 & 3  \\ \hline                    
    \end{tabular}
  \end{table}
\end{document}
%%% Local Variables:
%%% mode: latex
%%% TeX-master: t
%%% End:
