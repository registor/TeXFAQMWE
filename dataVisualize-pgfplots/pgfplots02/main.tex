% generates a 2D voronoi diagram from input files in tikz
\documentclass{article}
\pagestyle{empty}
\usepackage{tikz}
\usepackage{datatool}
\usepackage{graphicx}
\DTLloaddb[noheader,keys={idx,species,x,y,r}]{data}{data2Dtest.csv} %CSV data
\DTLloaddb[noheader,keys={one,two,three}]{tri}{tri2Dtest.csv} %delaunay triangulation data
\DTLloaddb[noheader,keys={vxx,vxy,vyx,vyy}]{vor}{vor2Dtest.csv} %voronoi data
\usetikzlibrary{calc,fadings}
%To save output, uncomment next three lines (Third only to force recompile).
%Then execute with pdflatex -shell-escape voronoi2D.tex
%\usetikzlibrary{external}
%\tikzexternalize % activate!
%\tikzset{external/force remake}

\begin{document}
%\begin{figure}
%\resizebox{\textwidth}{!}{

\begin{tikzpicture}[scale=0.5]
%Draw Atoms
\DTLforeach*{data}{\idx=idx, \species=species, \x=x, \y=y, \r=r}{
    %Choose color based on species, draw circle at current point
    \ifthenelse{\pdfstrcmp{\species}{Al}=0}{
    \shade[ball color=gray] (\x,\y) circle(0.4*\r);
    }{
    \shade[ball color=red] (\x,\y) circle(0.4*\r);
    }
}

%Draw Delaunay Simlpexes
\DTLforeach*{tri}{\one=one, \two=two, \three=three}{
    %One, two and three are indexes of the corners of each D triangle
    %Get locations from \x & \y in data db for each index
    \DTLgetvalueforkey{\oneX}{x}{data}{idx}{\one}
    \DTLgetvalueforkey{\oneY}{y}{data}{idx}{\one}
    \DTLgetvalueforkey{\twoX}{x}{data}{idx}{\two}
    \DTLgetvalueforkey{\twoY}{y}{data}{idx}{\two}
    \DTLgetvalueforkey{\threeX}{x}{data}{idx}{\three}
    \DTLgetvalueforkey{\threeY}{y}{data}{idx}{\three}
    %Draw triangle
    \draw (\oneX,\oneY) -- (\twoX,\twoY) -- (\threeX,\threeY) -- cycle;
}

%Draw Bounding Box
%Grab max/min of coords
\DTLcomputebounds{data}{x}{y}{\minX}{\maxX}{\minY}{\maxY}
%Draw Box
\draw[gray, thick] ({\minX-1},{\maxX-1}) rectangle ({\minY+1},{\maxY+1});
%Set clipping path for voronoi data
\clip ({\minX-1},{\maxX-1}) rectangle ({\minY+1},{\maxY+1});

%Draw Voronoi cells
\DTLforeach*{vor}{\vxx=vxx, \vxy=vxy, \vyx=vyx, \vyy=vyy}{
    %Draw line
    \draw[blue,dashed] (\vxx,\vyx) -- (\vxy,\vyy);
}
\end{tikzpicture}
%\end{figure}

\end{document}

%%% Local Variables:
%%% mode: latex
%%% TeX-master: t
%%% End:
