% 使用ctexart文档类(用XeLaTeX编译,直接支持中文)
\documentclass{standalone}

%导言区,可以在此引入必要的宏包
\usepackage{ctex}
\usepackage{tikz} % Required for drawing custom shapes
%\usetikzlibrary{arrows,calc,fit,matrix,positioning,shapes,shadows,trees,mindmap,tikzmark,arrows.meta}
\usetikzlibrary{decorations.text}

\begin{document} %在document环境中撰写文档

\begin{tikzpicture}[
  text=black,
  border/.style={line width=45mm},
  every node/.style={align=center},
  pin distance=17mm,
  ]
  % 尺寸宏命令
  \def\bgr{2.5cm}
  \def\sepr{3.65cm}
  \def\etxtr{3.8cm}
  \def\otxtr{2.5cm}
  \def\itxtr{1.2cm}
  \def\txtoff{4.5mm}
  
  % 绘制
  \foreach \etxto/\etxti/\txto/\txti/\name/\angle/\col [remember=\angle as \last (initially 30)] in
  { {搜索引擎/工具/策略等}/{网络/数字化资源}/{获取资源}/探究知识/信息/90/blue!21,
    {文字/图像/数据处理等}/{复制/粘贴功能}/提高效率/支持构建/效率/150/cyan!33,
    {在线批阅/电子档案等}/{反馈/反思}/{记录/展示}/促进反思/评价/210/green!22,
    {专家系统/思维导图等}/{独特形式再现知识}/高阶学习/高阶思维/认识/270/red!21,
    {Email/论坛/聊天等}/{异步/同步交流}/协作学习/{}/交流/330/orange!21,
    {案例/问题/项目等}/{情境模拟/游戏化}/创建情境/做中学/情境/390/violet!21}
  {
    % 绘制背景
    \draw[\col, border] (\last:\bgr) arc[start angle=\last, end
    angle=\angle, radius=\bgr];
    % 不同扇形角间的分割条    
    \draw[white, line width=2mm] (\last:1.1cm)--++(\last:\sepr);

    % 绘制工具示例外层文字
    \path [postaction={decorate, decoration={text
        align={center},raise={\txtoff},text along path,
        text={\etxto}}}](\angle:\etxtr) arc[start angle=\angle, end
    angle=\last, radius=\etxtr];

    % 绘制工具示例内层文字
    \path [postaction={decorate, decoration={text
        align={center},raise={0.0mm},text along path,
        text={\etxti}}}](\angle:\etxtr) arc[start angle=\angle, end
    angle=\last, radius=\etxtr];

    % 绘制工具内涵外层文字
    \path [postaction={decorate, decoration={text
        align={center},raise={\txtoff},text along path,
        text={\txto}}}](\angle:\otxtr) arc[start angle=\angle, end
    angle=\last, radius=\otxtr];

    % 绘制工具内涵内层文字
    \path [postaction={decorate, decoration={text
        align={center},raise={0.0mm},text along path,
        text={\txti}}}](\angle:\otxtr) arc[start angle=\angle, end
    angle=\last, radius=\otxtr];

    % 绘制工具名称文字
    \path [postaction={decorate, decoration={text
        align={center},raise={\txtoff},text along path,
        text={\name}}}](\angle:\itxtr) arc[start angle=\angle, end
    angle=\last, radius=\itxtr];
    % 绘制工具两个字
    \path [postaction={decorate, decoration={text
        align={center},raise={0.0mm},text along path, text={工具}}}]
          (\angle:\itxtr) arc[start angle=\angle, end
    angle=\last, radius=\itxtr];
  }
  % 最后一个扇形分割条
  \draw[white, line width=2mm] (30:1.1cm)--++(30:\sepr);
  % 中心图
  \node[line width=1.2mm, draw, circle, minimum width=5em, white,
  fill=yellow!51,text=black] (core) {学习工具};
  % 分割圆
  \draw[red, dashed, line width = 0.5mm] (0,0) circle(2.2cm);
  \draw[blue, dashed, line width = 0.5mm] (0,0) circle(3.5cm);
\end{tikzpicture}
\end{document}



%%% Local Variables:
%%% mode: latex
%%% TeX-master: t
%%% End:
