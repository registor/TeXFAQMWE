\documentclass[12pt, border = 8pt, varwidth, convert]{standalone}

% 中文支持
\usepackage{ctex}

% 通过数据表/文件排版表格的宏包
\usepackage{pgfplotstable}
% pgfplotstable推荐加载的宏包
\usepackage{booktabs}
\usepackage{array}
\usepackage{colortbl}
% 生成文件
\usepackage{filecontents}
% 生成逗号分隔的数据文件
% #或%表示注释
% 如果数据中有逗号,用{}进行分组,如{grad(log(dof),log(error2))}
\begin{filecontents}{example.csv}
# Convergence results
# fictional source generated 2008
level,dof,error1,error2,info,{grad(log(dof),log(error2))},quot(error1)
1,9,2.50000000e-01,7.57858283e-01,48,0,0
2,25,6.25000000e-02,5.00000000e-01,25,-1.35691545e+00,4
3,81,1.56250000e-02,2.87174589e-01,41,-1.17924958e+00,4
4,289,3.90625000e-03,1.43587294e-01,8,-1.08987331e+00,4
5,1089,9.76562500e-04,4.41941738e-02,22,-1.04500712e+00,4
6,4225,2.44140625e-04,1.69802322e-02,46,-1.02252239e+00,4
7,16641,6.10351562e-05,8.20091159e-03,40,-1.01126607e+00,4
8,66049,1.52587891e-05,3.90625000e-03,48,-1.00563427e+00,3.99999999e+00
9,263169,3.81469727e-06,1.95312500e-03,33,-1.00281745e+00,4.00000001e+00
10,1050625,9.53674316e-07,9.76562500e-04,2,-1.00140880e+00,4.00000001e+00
\end{filecontents}

% 全局格式设置,逗号分隔各参数
\pgfplotstableset{
  col sep=comma,%数据用逗号分隔
  fixed zerofill,%固定小数位数,不足用0填充
  precision=3%精度
}

\begin{document}

\begin{table}[htp]
  \centering
  \caption{读入csv数据文件排版表格}
  \pgfplotstabletypeset{example.csv}  
\end{table}

\begin{table}[htp]
  \centering
  \caption{通过列名称/编号选择指定列}
  \pgfplotstabletypeset[columns={dof,level,[index]4}]{example.csv}  
\end{table}

\begin{table}[htp]
  \centering
  \caption{设置表格式}  
  \pgfplotstabletypeset[% 如果需要全局设置,可将这些格式在导言区用\pgfplotstableset设置
    % 通过columns/<colname>/.style={<options>}设置指定 <colname> 列的格式
    columns/dof/.style={int detect,column type=r,column name=\textsc{Dof}},
    columns/error1/.style={
      sci, sci zerofill, sci sep align, precision=1, sci superscript,
      column name=$e_1$,
    },
    columns/error2/.style={
      sci,sci zerofill,sci sep align,precision=2,sci 10e,
      column name=$e_2$,
    },
    columns/{grad(log(dof),log(error2))}/.style={
      string replace={0}{}, % 删除 '0'
      column name={$\nabla e_2$},
      dec sep align,
    },
    columns/{quot(error1)}/.style={
      string replace={0}{}, % 删除 '0'
      column name={$\frac{e_1^{(n)}}{e_1^{(n-1)}}$}
    },
    empty cells with={--}, % 替换空白单元格为'--'
    every head row/.style={% 标题行格式
      before row=\toprule,% 行前命令
      after row=\midrule% 行后命令
    },
    every last row/.style={% 最后一行格式
      after row=\bottomrule% 行后命令
    },
    1000 sep={\,},% 千分位分隔符
    columns/info/.style={
      fixed,fixed zerofill,precision=1,showpos,
      column type=r,
    }
    ]{example.csv}% 数据文件
\end{table}

% 将读入的数据存入宏命令
\pgfplotstableread{example.csv}\loadedtable
\begin{table}[htp]
  \centering
  \caption{多次使用预读csv数据文件}
  % 选择指定的列
  \pgfplotstabletypeset[columns={dof,error1}]\loadedtable
  \hspace*{2cm}
  \pgfplotstabletypeset[columns={dof,error2}]\loadedtable
\end{table}
  
\begin{table}[htp]
  \centering
  \caption{使用内嵌数据排版}
  % 选择指定的行
  \pgfplotstabletypeset[
    col sep=space,% 空格分隔各列
    skip first n=4,% 跳过前4行
    % 三线表
    every head row/.style={% 标题行格式
      before row=\toprule,% 行前命令
      after row=\midrule% 行后命令
    },
    every last row/.style={% 最后一行格式
      after row=\bottomrule% 行后命令
    },
    ]{%<- 必须使用这个'%',否则此处的newline会被认为是一个空行
      XYZ Format,
      Version 1.234
      Date 2010-09-01
      @author Mustermann
      A B C
      1 2 3
      4 5 6
    }
\end{table}

\begin{table}[htp]
  \centering
  \caption{设置彩色列}
  \pgfplotstabletypeset[
    columns={dof,error1,{grad(log(dof),log(error2))},info},% 列选择
    % 列格式设置
    columns/error1/.style={
      column name=$L_2$,
      sci,sci zerofill,sci subscript,
      precision=3
    },
    columns/dof/.style={
      int detect,
      column name=\textsc{Dof}
    },
    columns/{grad(log(dof),log(error2))}/.style={
      column name=slopes $L_2$,
      fixed,fixed zerofill,
      precision=1
    },
    % 每偶数列设置格式(从0开始计数)
    every even column/.style={
      column type/.add={>{\columncolor[gray]{.8}}}{}
    },
    % 三线表
    every head row/.style={% 标题行格式
      before row=\toprule,% 行前命令
      after row=\midrule% 行后命令
    },
    every last row/.style={% 最后一行格式
      after row=\bottomrule% 行后命令
    },
  ]\loadedtable%预读入的数据  
\end{table}

\begin{table}[htp]
  \centering
  \caption{设置彩色行}
  \pgfplotstabletypeset[
    columns={dof,error1,{grad(log(dof),log(error2))},info},% 列选择
    % 设置寄数行颜色
    every even row/.style={
      before row={\rowcolor[gray]{0.9}}
    },
    % 三线表
    every head row/.style={
      before row=\toprule,
      after row=\midrule
    },
    every last row/.style={
      after row=\bottomrule},
  ]\loadedtable%预读入的数据  
\end{table}  

\verb|pgfplotstable|宏包采用``key-value-options''方式设置各个参数,并
提供了丰富的格式控制,其细节,请使用\verb|texdoc pgfplotstable|参阅说
明书。  

\end{document}

%%% Local Variables:
%%% mode: latex
%%% TeX-master: t
%%% End:
