\documentclass[margin=3pt,
  varwidth, 
  convert,
  convert={
    outext=.png,
    command=\unexpanded{
      pdftocairo -r 300 -png \infile % convert pdf to png image
    }
  }
]{standalone}

\usepackage{tikz}
\usepackage{xcolor}
\input{pgfmanual-en-macros.tex}

\begin{document}

\begin{command}{\pgfdeclareimage\oarg{options}\marg{image name}\marg{filename}}
    Declares an image, but does not paint anything. To draw the image, use
    |\pgfuseimage{|\meta{image name}|}|. The \meta{filename} may not have an
    extension.  For \textsc{pdf}, the extensions |.pdf|, |.jpg|, and |.png|
    will automatically tried. For PostScript, the extensions |.eps|, |.epsi|,
    and |.ps| will be tried.

    The following options are possible:
    %
    \begin{itemize}
        \item \declare{|height=|\meta{dimension}} sets the height of the image.
            If the width is not specified simultaneously, the aspect ratio of
            the image is kept.
        \item \declare{|width=|\meta{dimension}} sets the width of the image.
            If the height is not specified simultaneously, the aspect ratio of
            the image is kept.
        \item \declare{|page=|\meta{page number}} selects a given page number
            from a multipage document. Specifying this option will have the
            following effect: first, \pgfname\ tries to find a file named
            %
            \begin{quote}
                \meta{filename}|.page|\meta{page number}|.|\meta{extension}
            \end{quote}
            %
            If such a file is found, it will be used instead of the originally
            specified filename. If not, \pgfname\ inserts the image stored in
            \meta{filename}|.|\meta{extension} and if a recent version of
            |pdflatex| is used, only the selected page is inserted. For older
            versions of |pdflatex| and for |dvips| the complete document is
            inserted and a warning is printed.
        \item \declare{|interpolate=|\meta{true or false}} selects whether the
            image should be ``smoothed'' when zoomed. False by default.
        \item \declare{|mask=|\meta{mask name}} selects a transparency mask.
            The mask must previously be declared using |\pgfdeclaremask| (see
            below). This option only has an effect for |pdf|. Not all viewers
            support masking.
    \end{itemize}
    %
\begin{codeexample}[code only]
\pgfdeclareimage[interpolate=true,height=1cm]{image1}{brave-gnu-world-logo}
\pgfdeclareimage[interpolate=true,width=1cm,height=1cm]{image2}{brave-gnu-world-logo}
\pgfdeclareimage[interpolate=true,height=1cm]{image3}{brave-gnu-world-logo}
\end{codeexample}
    %
\end{command}

% \begin{codeexample}[code only]
% \pgfdeclareimage[interpolate=true,height=1cm]{image1}{brave-gnu-world-logo}
% \pgfdeclareimage[interpolate=true,width=1cm,height=1cm]{image2}{brave-gnu-world-logo}
% \pgfdeclareimage[interpolate=true,height=1cm]{image3}{brave-gnu-world-logo}
% \end{codeexample}

\end{document}

%%% Local Variables:
%%% mode: latex
%%% TeX-master: t
%%% End:
