\documentclass{ctexart}

% 通过数据表/文件排版表格的宏包
\usepackage{pgfplotstable}
% pgfplotstable推荐加载的宏包
\usepackage{booktabs}
\usepackage{array}
\usepackage{colortbl}
\usepackage{longtable}
\usepackage{multirow}
% 生成文件
\usepackage{filecontents}
% 生成逗号分隔的数据文件
% #或%表示注释
% 如果数据中有逗号,用{}进行分组,如{grad(log(dof),log(error2))}
\begin{filecontents}{example.csv}
cola,colb,colc,cold,cole,colf,colg
CG.A.2,23.05,0.002,0.116,0.035,0.589,32491
CG.A.4,15.06,0.003,0.067,0.021,0.351,18211
CG.A.8,13.38,0.004,0.072,0.023,0.210,9890
CG.B.2,867.45,0.002,0.864,0.232,3.256,228562
CG.B.4,501.61,0.003,0.438,0.136,2.075,123862
CG.B.8,384.65,0.004,0.457,0.108,1.235,63777
MG.A.2,112.27,0.002,0.846,0.237,3.930,236473
MG.A.4,59.84,0.003,0.442,0.128,2.070,123875
MG.A.8,31.38,0.003,0.476,0.114,1.041,60627
MG.B.2,526.28,0.002,0.821,0.238,4.176,236635
MG.B.4,280.11,0.003,0.432,0.130,1.706,123793
MG.B.8,148.29,0.003,0.442,0.116,0.893,60600
LU.A.2,2116.54,0.002,0.110,0.030,0.532,28754
LU.A.4,1102.50,0.002,0.069,0.017,0.255,14915
LU.A.8,574.47,0.003,0.067,0.016,0.192,8655
LU.B.2,9712.87,0.002,0.357,0.104,1.734,101975
LU.B.4,4757.80,0.003,0.190,0.056,0.808,53522
LU.B.8,2444.05,0.004,0.222,0.057,0.548,30134
CG.B.2,867.45,0.002,0.864,0.232,3.256,228562
CG.B.4,501.61,0.003,0.438,0.136,2.075,123862
CG.B.8,384.65,0.004,0.457,0.108,1.235,63777
MG.A.2,112.27,0.002,0.846,0.237,3.930,236473
MG.A.4,59.84,0.003,0.442,0.128,2.070,123875
MG.A.8,31.38,0.003,0.476,0.114,1.041,60627
MG.B.2,526.28,0.002,0.821,0.238,4.176,236635
MG.B.4,280.11,0.003,0.432,0.130,1.706,123793
MG.B.8,148.29,0.003,0.442,0.116,0.893,60600
LU.A.2,2116.54,0.002,0.110,0.030,0.532,28754
LU.A.4,1102.50,0.002,0.069,0.017,0.255,14915
LU.A.8,574.47,0.003,0.067,0.016,0.192,8655
LU.B.2,9712.87,0.002,0.357,0.104,1.734,101975
LU.B.4,4757.80,0.003,0.190,0.056,0.808,53522
LU.B.8,2444.05,0.004,0.222,0.057,0.548,30134
EP.A.2,123.81,0.002,0.010,0.003,0.074,1834
EP.A.4,61.92,0.003,0.011,0.004,0.073,1743
EP.A.8,31.06,0.004,0.017,0.005,0.073,1661
EP.B.2,495.49,0.001,0.009,0.003,0.196,2011
EP.B.4,247.69,0.002,0.012,0.004,0.122,1663
EP.B.8,126.74,0.003,0.017,0.005,0.083,1656
\end{filecontents}

% 全局格式设置,对所有表格有效
\pgfplotstableset{
  col sep=comma,%数据用逗号分隔
  fixed zerofill,%固定小数位数,不足用0填充
  precision=3%精度
}

% 定义跨页表表头
\newcommand{\headrow}{
   \multirow{2}{*}{测试程序} & \multicolumn{1}{c}{正常运行} &
   \multicolumn{1}{c}{同步} & \multicolumn{1}{c}{检查点} &
   \multicolumn{1}{c}{卷回恢复} & \multicolumn{1}{c}{进程迁移} &
   \multicolumn{1}{c}{检查点} \\
   & \multicolumn{1}{c}{时间(s)} & \multicolumn{1}{c}{时间(s)} &
   \multicolumn{1}{c}{时间(s)} & \multicolumn{1}{c}{时间(s)} &
   \multicolumn{1}{c}{时间(s)} & \multicolumn{1}{c}{文件(KB)} \\
 }
 % 定义跨页表续表表题
 \newcommand{\ctncap}{
   \multicolumn{7}{c}{续表~\thetable\hskip1em 实验数据}  
 }

\begin{document}

% \begin{table}[htp]
%   \centering
%   \caption{设置表格式}  
  \pgfplotstabletypeset[
    % 使用longtable环境
    begin table=\begin{longtable},% 
    end table=\end{longtable},%
    % 设置列格式(通过索引号选择列)
    display columns/0/.style={%
      column name={\multirow{2}{*}{测试程序}},% 列名称
      column type={c},% 列格式
      string type% 数据类型
    },%
    display columns/1/.style={%
      column name=正常运行,% 列名称
      column type={r},% 列格式
      fixed zerofill,%固定小数位数,不足用0填充
      precision=2%精度
    },%
    display columns/2/.style={%
      column name=同步,% 列名称
      column type={r},% 列格式
      fixed zerofill,%固定小数位数,不足用0填充
      precision=3%精度
      %int detect%
    },%
    display columns/3/.style={%
      column name=检测点,% 列名称
      column type={r},% 列格式
      %int detect% 数据类型
    },%
    display columns/4/.style={%
      column name=卷回恢复,% 列名称
      column type={r},% 列格式
      %int detect% 数据类型
    },%
    display columns/5/.style={%
      column name=进程迁移,% 列名称
      column type={r},% 列格式
      int detect% 数据类型
    },%
    display columns/6/.style={%
      column name=检查点,% 列名称
      column type={r},% 列格式
      int detect% 数据类型
    },%      
    every head row/.append style={
      before row={\caption[实验数据]{实验数据,这个题注十分的长,注意这在索引中的处理方式,还有 caption 后面的双反斜杠\label{tabpgfltab}}\\\toprule},
      after row={%          
        & 时间(s) & 时间(s) & 时间(s) & 时间(s) & 时间(s) & 文件(KB)\\ % 数据单位,用 & 分割
        \midrule
        \endfirsthead
        \ctncap\\
        \toprule
        \headrow
        \midrule
        \endhead
        \hline
        \multicolumn{7}{r}{续下页} \\
        \endfoot
        \endlastfoot
      },
    },
    every last row/.style={after row=\bottomrule}, %
  % %]{db4.csv} % filename/path to file
    ]{example.csv}% 数据文件
%\end{table}

\verb|pgfplotstable|宏包采用``key-value-options''方式设置各个参数,并
提供了丰富的格式控制,其细节,请使用\verb|texdoc pgfplotstable|参阅说
明书。  

\end{document}

%%% Local Variables:
%%% mode: latex
%%% TeX-master: t
%%% End:
