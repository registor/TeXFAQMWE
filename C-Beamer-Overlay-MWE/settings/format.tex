% ====================================================================================
% bytefield绘图设置
% 调整表格的垂直对齐方式
\newcommand{\descbox}[1]{\parbox[c][0.7\baselineskip]{0.95\width}{%
\raggedright #1\vfill}}
% 用于绘制无标记内存图中一个内存块的命令,其开始地址在底部,结束地址在顶部(基于bytefield宏包)
% 语法:
% \memsec{结束地址}{开始地址}{以行为单位的高度}{盒子中的文字}
\newcommand*\istempaddr[1]{\IfStrEq{}{#1}{}{0X#1}}
\newcommand{\memsec}[4]{%
  % 定义memsection的高度
  %
  %\tiny
  \bytefieldsetup{bitheight=#3\baselineskip}%
  \bitbox[]{5}{%    
    \tiny \texttt{\istempaddr{#1}}%0X#1}% 结束地址
    \\
    % 留出空白
    \vspace{#3\baselineskip}
    \vspace{-1.6\baselineskip}
    \vspace{-#3pt}
    \tiny \texttt{\istempaddr{#2}}%0X#2}% 开始地址
  }%
  \bitbox{4}{\tiny \texttt{#4}}% 盒子里的内容 %\vspace{-0.9\baselineskip}
}

% 定义绘制有标记内存的命令(基于bytefield宏包)
% 创建计算机内存映像的自定义命令,
% 起始地址位于低部,结束地址位于顶部
% 语法:
% \memsection[标记符号]{结束地址}{起始地址}{内存单元高度}{内存单元中的内容}
% 其中标记符号为可选项,可以是任何符合LaTeX语法的内容
\newcommand{\memseclab}[5][]{
  %\tiny
  \bytefieldsetup{bitheight=#4\baselineskip} % 内存单元的高度
  \bitbox[]{10}{%标记符号及地址盒子,参数为总宽度
    \raggedleft%右对齐
    \texttt{\tiny\istempaddr{#2}}% 结束地址
    \\%换行
    % 设置结束地址与起始地址之间的空白
    \vspace{#4\baselineskip}
    \vspace{-1.5\baselineskip}
    \vspace{-#4pt} % 
    {\tiny\texttt{#1}}%\hspace{1em}
    \texttt{\tiny\istempaddr{#3}}% 起始地址
  }~
  \bitbox{4}{\tiny \texttt{#5}} % 内存单元盒子
}

% 自定义垂直省略号
\makeatletter
\DeclareRobustCommand{\rvdots}{%
  \vbox{
    \baselineskip3\p@\lineskiplimit\z@
    \kern-\p@
    \hbox{.}\hbox{.}\hbox{.}
  }}
\makeatother
% ====================================================================================

% ====================================================================================
% 重定义强调字体的代码,解决默认强调字体是italic,此时中文会用楷体代替,
% 在此设置为加粗,注意需要使用etoolbox宏包
\makeatletter
\let\origemph\emph
\newcommand*\emphfont{\normalfont\bfseries}
\DeclareTextFontCommand\@textemph{\emphfont}
\newcommand\textem[1]{%
  \ifdefstrequal{\f@series}{\bfdefault}
    {\@textemph{\CTEXunderline{#1}}}
    {\@textemph{#1}}%
}
\RenewDocumentCommand\emph{s o m}{%
  \IfBooleanTF{#1}
    {\textem{#3}}
    {\IfNoValueTF{#2}
      {\textem{#3}\index{#3}}
      {\textem{#3}\index{#2}}%
     }%
}
\makeatother   
% ====================================================================================

% 代码显示模式设置(基于minted宏包)
% ====================================================================================
\usemintedstyle{default}  %codeblocks模式

% 定义排版C语言代码命令
\setminted{fontsize=\tiny, breaklines=true,
  breakautoindent=false}
\newmintinline{c}{fontsize=\normalsize, autogobble}
\newmintinline[cinlinett]{c}{fontsize=\normalsize, escapeinside=||, autogobble}
\newmintinline[cinttscr]{c}{fontsize=\scriptsize, escapeinside=||, autogobble}
\newmintinline[cinttfts]{c}{fontsize=\footnotesize, escapeinside=||, autogobble}
\newmintinline[cintttny]{c}{fontsize=\tiny, escapeinside=||, autogobble}
\newmintinline[cinttlrg]{c}{fontsize=\large, escapeinside=||, autogobble}
\newminted{c}{frame=lines, autogobble}
\newminted[ctt]{c}{mathescape,frame=lines,escapeinside=||, autogobble}
\newmintedfile{c}{frame=lines}%linenos=true,
\newmintedfile[cfilett]{c}{frame=lines,escapeinside=||}%linenos=true,
\newmintedfile[cfttscr]{c}{fontsize = \scriptsize, frame=lines,escapeinside=||}%linenos=true,
\newmintedfile[cfiletikz]{c}{frame=lines}%linenos=true,
% ====================================================================================

% TikZ宏包扩展
% ====================================================================================
\usetikzlibrary{graphs}
\usetikzlibrary{mindmap,trees}
\tikzset{every concept/.style={minimum size=1cm, text width=1.5cm}}
\usetikzlibrary{shapes,arrows,chains}
\usetikzlibrary{positioning}
\usetikzlibrary{calc}
\usetikzlibrary{arrows.meta}
\usetikzlibrary{decorations.pathreplacing}
\usetikzlibrary{tikzmark}
\usetikzlibrary{shapes.geometric}
\usetikzlibrary{decorations.text}
\usetikzlibrary{matrix}
\usetikzlibrary{backgrounds}

% 流程图属性设置
% ====================================================================================
% 各参数默认值
\flowchartset{
  free color = green, % 自由连线颜色(默认取green)
  norm color = blue, % 常规连线颜色(默认取blue)
  cong color = red, % 关联连线颜色(默认取red)
  proc fill color = orange!10, % 顺序处理框填充颜色(默认取白色)
  test fill color = green!30, % 判断框填充颜色(默认取白色)
  io fill color = blue!30, % 输入/输出框填充颜色(默认取白色)
  term fill color = red!30, % 开始/结束框填充颜色(默认取白色)
  flow line width = 1.3\pgflinewidth, % 各类流程线线条宽度(默认取当前线条宽度)
}
% ====================================================================================

% 设置动画显示样式
% ====================================================================================
\tikzset{
  invisible/.style={opacity=0},
  visible on/.style={alt={#1{}{invisible}}},
  alt/.code args={<#1>#2#3}{%
    \alt<#1>{\pgfkeysalso{#2}}{\pgfkeysalso{#3}} % \pgfkeysalso doesn't change the path
  },
}
% ====================================================================================

%%%%%%%%%% 绘制链表数据结构需要的命令TiKz%%%%%%%%
% ====================================================================================
\usetikzlibrary{shapes,arrows,calc}
\newcommand{\data}{data \nodepart{second} \phantom{null}}

\tikzstyle{ptr}  = [draw, -latex', blue]
\tikzstyle{head} = [rectangle, draw, text height=3mm, text width=3mm, text centered, node distance=3cm, inner sep=0pt]
\tikzstyle{data} = [rectangle split, rectangle split parts=2, draw, text centered, minimum height=3em]
% ====================================================================================

% 改变脚注的符号
% ====================================================================================
\setbeamerfont{footnote}{size=\zihao{7}} % 改变脚注字号
\makeatletter
\def\@fnsymbol#1{\ensuremath{\ifcase#1\or *\or \dagger\or \ddagger\or
   \mathsection\or \mathparagraph\or \|\or **\or \dagger\dagger
   \or \ddagger\ddagger \else\@ctrerr\fi}}
\makeatother
\renewcommand{\thefootnote}{\fnsymbol{footnote}}
% ====================================================================================

% 定义C语言中的专有名词命令
% ====================================================================================
\newcommand{\cb}{\texttt{Code::Blocks}}
\newcommand{\mww}{\texttt{MinGW-w64}}
\newcommand{\mfile}{\qtmark{\texttt{Makefile}}}
\newcommand{\vs}{\texttt{Visual Studio}}
\newcommand{\gcc}{\texttt{gcc}}
\newcommand{\db}{\texttt{DEBUG}}
\newcommand{\dbger}{\texttt{Debugger}}
\newcommand{\gdb}{\texttt{gdb}调试器}
\newcommand{\cdb}{\texttt{cdb}调试器}
\newcommand{\gdbcmd}{\texttt{(gdb)}}
\newcommand{\scf}{\cinline[fontsize = \scriptsize]{scanf()}}
\newcommand{\ptf}{\cinline{printf()}}
\newcommand{\rn}{\cinline{'\n'}}
\newcommand{\cch}[1]{\cinline{'#1'}}
\newcommand{\lumos}{Linux、Unix、Mac OS}
\newcommand{\win}{Windows}
\newcommand{\unix}{UNIX}
\newcommand{\lnx}{Linux}

\newcommand{\cl}{{C语言}}
\newcommand{\nint}{{\cinline{int}}}
\newcommand{\flt}{{\cinline{float}}}
\newcommand{\snint}{{\cinline{short int}}}
\newcommand{\lint}{{\cinline{long int}}}
\newcommand{\llint}{{\cinline{long long int}}}
\newcommand{\ldb}{{\cinline{long double}}}
\newcommand{\sch}{{\cinline{signed char}}}
\newcommand{\sint}{{\cinline{signed int}}}
\newcommand{\ssint}{{\cinline{signed short int}}}
\newcommand{\slint}{{\cinline{signed long int}}}
\newcommand{\sllint}{{\cinline{signed long long int}}}
\newcommand{\usch}{{\cinline{unsigned char}}}
\newcommand{\usint}{{\cinline{unsigned int}}}
\newcommand{\ussint}{{\cinline{unsigned short int}}}
\newcommand{\uslint}{{\cinline{unsigned long int}}}
\newcommand{\usllint}{{\cinline{unsigned long long int}}}

% 路径设置
% ====================================================================================
\graphicspath{{figure/}}%图片所在的目录
% ====================================================================================


%%% Local Variables: 
%%% mode: latex
%%% TeX-master: "../main.tex"
%%% End: 
