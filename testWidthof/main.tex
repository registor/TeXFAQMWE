\documentclass[varwidth,convert]{standalone}

\usepackage{ctex}

% 计算文字宽度(\widthof命令)
\usepackage{calc}

% 使用\foreach命令
\usepackage{tikz}

% 读取数据文件
\usepackage{datatool}

% 记录字符串长度的变量
\newlength{\textlen}
\newlength{\lablen}
\setlength{\lablen}{0pt}

\newcommand\college{学院}
\newcommand\majorID{专业班级}
\newcommand\thesisauthor{学生姓名}
\newcommand\adviser{指导教师}
\newcommand\coadviser{协助指导教师}
\newcommand\applydate{完成日期}

\begin{document} %在document环境中撰写文档

% 找到最长的字符串(用\settowidth命令实现)
\settowidth{\lablen}{\coadviser} % 不需要calc宏包
% 用\makebox生成分散对齐的定宽盒子
\begin{tabular} {cc}%
  \makebox[\the\lablen][s]{\college} :     & \makebox[\the\lablen][s]{} \\ \cline{2-2}%
  \makebox[\the\lablen][s]{\majorID} :     & \makebox[\the\lablen][s]{} \\ \cline{2-2}
  \makebox[\the\lablen][s]{\thesisauthor} :& \makebox[\the\lablen][s]{} \\ \cline{2-2}
  \makebox[\the\lablen][s]{\adviser} :     & \makebox[\the\lablen][s]{} \\ \cline{2-2}
  \makebox[\the\lablen][s]{\coadviser} :   & \makebox[\the\lablen][s]{} \\ \cline{2-2}
  \makebox[\the\lablen][s]{\applydate} :   & \makebox[\the\lablen][s]{} \\ \cline{2-2}
\end{tabular}

\vspace{4ex}

% 找到最长的字符串(用\widthof命令实现)
\setlength{\lablen}{\widthof{\coadviser}} % 需要calc宏包
% 用\makebox生成分散对齐的定宽盒子
\begin{tabular} {cc}%
  \makebox[\the\lablen][s]{\college} :     & \makebox[\the\lablen][s]{} \\ \cline{2-2}%
  \makebox[\the\lablen][s]{\majorID} :     & \makebox[\the\lablen][s]{} \\ \cline{2-2}
  \makebox[\the\lablen][s]{\thesisauthor} :& \makebox[\the\lablen][s]{} \\ \cline{2-2}
  \makebox[\the\lablen][s]{\adviser} :     & \makebox[\the\lablen][s]{} \\ \cline{2-2}
  \makebox[\the\lablen][s]{\coadviser} :   & \makebox[\the\lablen][s]{} \\ \cline{2-2}
  \makebox[\the\lablen][s]{\applydate} :   & \makebox[\the\lablen][s]{} \\ \cline{2-2}
\end{tabular}

\vspace{4ex}

% 找到最长的字符串(用循环实现,需要TikZ宏包)
\foreach \x in {\college, \majorID, \thesisauthor, \adviser, \coadviser, \applydate}
{
  \settowidth{\textlen}{\x}
  \ifdim \textlen > \lablen
    \setlength{\lablen}{\textlen}
  \else
    \relax
  \fi
}
% 用\makebox生成分散对齐的定宽盒子
\begin{tabular} {cc}%
  \makebox[\the\lablen][s]{\college} :     & \makebox[\the\lablen][s]{} \\ \cline{2-2}%
  \makebox[\the\lablen][s]{\majorID} :     & \makebox[\the\lablen][s]{} \\ \cline{2-2}
  \makebox[\the\lablen][s]{\thesisauthor} :& \makebox[\the\lablen][s]{} \\ \cline{2-2}
  \makebox[\the\lablen][s]{\adviser} :     & \makebox[\the\lablen][s]{} \\ \cline{2-2}
  \makebox[\the\lablen][s]{\coadviser} :   & \makebox[\the\lablen][s]{} \\ \cline{2-2}
  \makebox[\the\lablen][s]{\applydate} :   & \makebox[\the\lablen][s]{} \\ \cline{2-2}
\end{tabular}

% 使用datatool宏包从csv数据文件中读取数据,计算长度后进行排版
\DTLloaddb{lablist}{lablist.csv} % 打开数据文件
% 通过循环查找最长字符串的长度
\DTLforeach{lablist}{
  % .csv 档里对应每一列的宏
  \Name=lab%
}{
  \settowidth{\textlen}{\Name}
  \ifdim \textlen > \lablen
    \setlength{\lablen}{\textlen}
  \else
    \relax
  \fi
}

% 通过循环将字符串分散对齐
\begin{center}
  \DTLforeach{lablist}{
  % .csv 档里对应每一列的宏
  \Name=lab%
  }{
    \makebox[\the\lablen][s]{\Name}:\uline{\makebox[\the\lablen][c]{}}\\
  }
\end{center}
\end{document}

%%% Local Variables:
%%% mode: latex
%%% TeX-master: t
%%% End:
