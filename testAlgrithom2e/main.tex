\documentclass{ctexart}
\usepackage{algorithm2e}
\usepackage{xfrac} % \sfrac斜分式命令

% 框线颜色设置
\usepackage{xcolor}
\makeatletter
\newcommand{\setalgotoprulecolor}[1]{\colorlet{toprulecolor}{#1}}
\let\old@algocf@pre@ruled\@algocf@pre@ruled % Adjust top rule colour
\renewcommand{\@algocf@pre@ruled}{\textcolor{toprulecolor}{\old@algocf@pre@ruled}}

\newcommand{\setalgobotrulecolor}[1]{\colorlet{bottomrulecolor}{#1}}
\let\old@algocf@post@ruled\@algocf@post@ruled % Adjust middle rule colour
\renewcommand{\@algocf@post@ruled}{\textcolor{bottomrulecolor}{\old@algocf@post@ruled}}

\newcommand{\setalgomidrulecolor}[1]{\colorlet{midrulecolor}{#1}}
\renewcommand{\algocf@caption@ruled}{%
  \box\algocf@capbox{\color{midrulecolor}\kern\interspacetitleruled\hrule
    width\algocf@ruledwidth height\algotitleheightrule depth0pt\kern\interspacealgoruled}}
\makeatother

\setalgotoprulecolor{blue!90}% Default
\setalgobotrulecolor{red!90}% Default
\setalgomidrulecolor{green!90}% Default

\begin{document}
计算分数累加和的for循环实现如算法\ref{algo:sumwithfor}所示。

\begin{algorithm}[htp]
  \SetAlgoLined % 绘制区域垂直线
  \LinesNumbered % 排版行号
  \KwIn{$x_i=\{1,\sfrac{1}{2},\sfrac{1}{3},\sfrac{1}{4},\sfrac{1}{5},\cdots,\sfrac{1}{100}\}$}
  \KwOut{$sum=\sum\limits_{i=1}^{100}  x_i$}              
  \BlankLine  % 空白线
  Initialize $sum=0$\;
  \For{$i\leftarrow1$ \KwTo $100$}{
    $sum=sum+\sfrac{1}{i}$\;
    $i=i+1$\;
  }
  Output $sum$\;
  \caption{sum of frac}\label{algo:sumwithfor} % 标题
\end{algorithm}

计算分数累加和的while循环实现如算法\ref{algo:sumwithwhile}所示。

\RestyleAlgo{ruled} % 设置排版样式
% 设置文字关键字
\SetAlgorithmName{算法}{算法}{算法列表}
\SetKwInput{KwIn}{输入}
\SetKwInput{KwOut}{输出}
\begin{algorithm}[htp]
  \KwIn{$x_i=\{1,\sfrac{1}{2},\sfrac{1}{3},\sfrac{1}{4},\sfrac{1}{5},\cdots,\sfrac{1}{100}\}$}
  \KwOut{$sum=\sum\limits_{i=1}^{100}  x_i$}              
  \AlgoDisplayBlockMarkers
  \SetAlgoBlockMarkers{begin}{end}%
  \SetAlgoNoEnd
  \SetKwFunction{FSum}{getSum}%
  \SetKwProg{Fn}{}{}{}
  \Fn{}{
    $sum = 0$\;
    $i = 1$\;
    \While{i <= 100}{
      $sum = sum + \sfrac{1}{i}$\;
      $i = i + 1$\;
    }     
    print $sum$\;
  }
  \caption{计算分数累加和}\label{algo:sumwithwhile}
\end{algorithm}
\end{document}

%%% Local Variables:
%%% mode: latex
%%% TeX-master: t
%%% End:
