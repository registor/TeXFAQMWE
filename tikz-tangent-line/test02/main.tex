% 摘录自https://tex.stackexchange.com/questions/197704/aligning-tangents-of-graphs/198046#198046
\documentclass[border=5mm]{standalone}  
\usepackage{pgfplots}
\pgfplotsset{compat=1.10}
\usetikzlibrary{intersections}

\makeatletter
\def\parsenode[#1]#2\pgf@nil{%
    \tikzset{label node/.style={#1}}
    \def\nodetext{#2}
}

\tikzset{
    add node at x/.style 2 args={
        name path global=plot line,
        /pgfplots/execute at end plot visualization/.append={
                \begingroup
                \@ifnextchar[{\parsenode}{\parsenode[]}#2\pgf@nil
            \path [draw,name path global = position line #1-1]
                ({axis cs:#1,0}|-{rel axis cs:0,0}) --
                ({axis cs:#1,0}|-{rel axis cs:0,1});
            \path [draw,xshift=1pt, name path global = position line #1-2]
                ({axis cs:#1,0}|-{rel axis cs:0,0}) --
                ({axis cs:#1,0}|-{rel axis cs:0,1});
            \path [
                name intersections={
                    of={plot line and position line #1-1},
                    name=left intersection
                },
                name intersections={
                    of={plot line and position line #1-2},
                    name=right intersection
                },
                label node/.append style={pos=1}
            ] (left intersection-1) -- (right intersection-1)
            node [label node]{\nodetext};
            \endgroup
        }
    }
}
\makeatother

\begin{document}
\begin{tikzpicture}
  \begin{axis}[%
      xmin = 0, xmax = 3.25,
      ymin = 0,
      domain=0:3,
      samples=50,
      tangent/.style={
            add node at x={#1}{
                [
                    sloped, 
                    append after command={(\tikzlastnode.west) edge [thick, red!75!black] (\tikzlastnode.east)},
                    minimum width=4cm
                ]
            }      
      }
      ]%
    \addplot [gray, tangent=1, tangent=2] {ln(3*(3-x)+1)};
    %\addplot [gray, tangent=1] {ln(9*x^2+6*x+1)};
    %\addplot [gray, tangent=1] {ln(9*x^2+6*x+1)+ln(3*(3-x)+1)};
  \end{axis}%
\end{tikzpicture}%
\end{document}

%%% Local Variables:
%%% mode: latex
%%% TeX-master: t
%%% End:
