% 使用ctexart文档类(用XeLaTeX编译,直接支持中文)
\documentclass{ctexart}

\usepackage{geometry}
\geometry{% 本科生页边距
  paper=a4paper,
  top=2.5cm,
  bottom=2.1cm,
  left=2.5cm,
  right=2.5cm,
  headheight=0.8cm,
  headsep=0.7cm,
  footskip=1.4cm,
} 

\usepackage{pgfplotstable}
\usepackage{multirow}

% 载入拼音宏包
\usepackage{xpinyin}
% 设置拼音与正文字号比例
\xpinyinsetup{ratio=0.5}

% 数据格式设置
\pgfplotstableset{
  col sep=comma,
  header=true,
  font=\small,
}

% 读入数据
\pgfplotstableread{namelist.csv}{\loadedtable}

\def\coursename{C语言程序设计}
\def\courseid{1091102}
\def\courseidx{2}
\def\teacher{耿楠}
\begin{document} %在document环境中撰写文档
\pagestyle{empty}
\def\arraystretch{1.2}
\pgfplotstabletypeset[
  % columns={stno,name,1,2},
  column type/.add={|c|}{},% results in ’|c’
  %column type/.add={|@{\extracolsep{0.5cm}}}{@{\extracolsep{0.5cm}}},% results in ’|c’
  every head row/.style={
    % patches the first row:
    before row={
      \multicolumn{\pgfplotstablecols}{c}{\bf\large 西北农林科技大学2019秋季学期学生课程考勤表}\\
      \multicolumn{\pgfplotstablecols}{l}{课程名称:\coursename\hfill
        课程代码:\courseid\hfill 课序号:\courseidx\hfill 任课教师:\teacher}\\
      \hline      
      \pgfmathtruncatemacro{\len}{\pgfplotstablecols-3}
      \global\let\colslen\len
      \multirow{2}{*}{学号}&\multirow{2}{*}{姓名}&\multirow{2}{*}{班级}&\multicolumn{\colslen}{c|}{平时考勤}\\
      \cline{4-\pgfplotstablecols}
    }, 
  },
  after row={\hline},
  every last row/.style={
    after row={
      \hline
      \multicolumn{16}{|l|}{备注:出勤:$\surd$\ 缺勤:$\times$\ 迟到:$\bigtriangleup$\ 早退:$\bigcirc$\ 事假:$\bigtriangledown$\ 病假:$\oplus$}\\
      \hline
    },
  },
  columns/stno/.style={string type,
    column name={},
    column type=|c,
  },
  columns/name/.style={string type,
    column name={},
    postproc cell content/.style={@cell content=\xpinyin*{##1}},
    column type=|c,
  },
  columns/class/.style={string type,
    column name={},
    column type=|c,
  },
  columns/notes/.style={
    column name={备注},
    column type=c,
  },
]{\loadedtable}

\end{document}

%%% Local Variables:
%%% mode: latex
%%% TeX-master: t
%%% End:
